\chapter{Hur påverkas ett team av sin arbetsmiljö?}
\chapterprecis{\LARGE{---- Hampus Dunström ----}}
\label{cha:indiv-report-hampus}

\section{Inledning}
\label{sec:introduction-hampus}

%% Skriv här

\subsection{Syfte}
\label{sec:purpose-hampus}

Att försöka skapa lite mer klarhet hur  arbetsmiljön påverkar mjukvaruutveckling i ett team. Genom att öka förståelsen av orsaker och konsekvenser av val inom arbetsmiljön för teamet så kan bättre val göras i framtiden.

\subsection{Frågeställningar}
\label{sec:issue-hampus}

\begin{itemize}
\item Hur påverkar arbetsmiljön kring ett team dess arbetssätt?
\item Hur påverkar arbetsmiljön sammanhållningen i teamet?
\item Hur påverkas gruppen av att få tillgång till ett eget rum där kandidatarbetet kan utföras jämfört med grupper som inte fått det?
\end{itemize}

\section{Bakgrund}
\label{sec:background-hampus}

Under studietiden har det genomförts en del projekt och grupparbeten alla med varierande resultat. Alla dessa projekt har haft olika förutsättningar i form av arbetsmiljö, vissa har utförts på olika platser varje gång man träffats andra på samma ställe varje gång. I vissa fall kan man lämna saker på sin arbetsplats ibland inte. Vissa grupparbeten har skett i högljudda miljöer andra i tysta. Vad har detta för betydelse för arbetets resultat och effektivitet? Denna undersökning görs i ett försök att ta reda på hur den optimala arbetsmiljön ser ut. 

\section{Teori}
\label{sec:theory-hampus}

%% Skriv här

\section{Metod}
\label{sec:method-hampus}

Det gjordes en empirisk studie i form av en enkät utskickad till de andra deltagare i kursen \textit{Kandidatprojekt i programvaruutveckling}. Enkäten skickades ut som ett Google Formulär med frågorna:
\begin{itemize}
\item Beskriv kort projektet du arbetar med.*
\item Hur intresserad av projektet är du?*
\item Hur många gånger i veckan träffas hela gruppen?*
\item Hur stor del av ditt arbete gör du ensam?
\item Föredrar du att jobba ensam eller tillsammans med någon?*
\item Vart brukar du utföra ditt arbete, hemma, i skolan eller...?*
\item Hur är ljudvolymen på dessa platser?
\item Hur är belysningen på dessa platser?, trivs du med den?
\item Är det folk runt omkring dig där, isåfall vad gör dom?
\item Har ni fått tillgång till någon lokal, isåfall hur ofta används den?*
\item Hur tycker du sammanhållningen i ditt team är?*
\item Varför tror du sammanhållningen är som den är?
\end{itemize}
Frågorna med '*' på slutet var obligatoriska. Svaren på enkäten kan ses i bilaga [BIFOGA SVAR!]. Det gjordes också litteraturstudier av liknande undersökningar och artiklar i ämnet som kunde hittas på internet. Utifrån denna data och de personliga erfarenheterna som samlats under utförandet av Kandidatprojektet gjordes sedan en diskussion och slutsatser drogs. Dessa finns beskrivna under Diskussion \ref{sec:discussion-hampus} och Slutsatser \ref{sec:conclusions-hampus}. 

Valet av både enkät och intervjuer gjordes för att både få kvantitativ data från enkäten och lite mer kvalitativ data från intervjuerna. Detta tillsammans med resultaten från litteraturstudien gav en bra bredd av datainsamling.

\section{Resultat}
\label{sec:results-hampus}

Det var 27 av de 98 studenter som läser kursen \textit{Kandidatprojekt i programvaruutveckling} som svarade på enkäten. Sammanfattade svar kan ses nedan:\\
\\\textbf{Hur intresserad av projektet är du?}\\ 
Genomsnittsintresset låg på 3.8 på en skala 1-5.\\
\textbf{Hur många gånger i veckan träffas hela gruppen?}\\
Grupperna träffades i genomsnitt 3.5 gånger per vecka.\\
\textbf{Hur stor del av ditt arbete gör du ensam?}\\
Större delen av arbetet gjordes i grupp, 16 studenter svarade 25\%-50\% 9 svarade  mindre än 25\% och 2 studenter svarade att de gjorde en större del än så ensamma.\\
\textbf{Föredrar du att jobba ensam eller tillsammans med någon?}\\
Det allra flesta studenterna föredrog att arbeta tillsammans med någon.\\
\textbf{Vart brukar du utföra ditt arbete, hemma, i skolan eller...?}\\
Skolan var den plats som de flesta gjorde större delen av sitt arbete. \\
\textbf{Hur är ljudvolymen på dessa platser?}\\
Större delen av de svarande tyckte ljudvolymen var låg eller bra.\\
\textbf{Hur är belysningen på dessa platser?, trivs du med den?}\\
Studenterna var nöjda med belysningen, flera uttryckte att de föredrog fönster med naturligt ljus och enstaka studenter uttryckte specifikt misstycke för lysrör.
\textbf{Är det folk runt omkring dig där, isåfall vad gör dom?}\\
Runt omkring arbetsplatsen svarade större delen att det var andra studenter som också arbetade med sina studier.\\
\textbf{Har ni fått tillgång till någon lokal, isåfall hur ofta används den?}\\
De flesta hade inte fått tillgång till någon specifik lokal men de som hade det använde den flitigt. Av de som inte fått någon lokal uttryckte flera att de bokade olika lokaler i skolan nästan varje dag.\\
\textbf{Hur tycker du sammanhållningen i ditt team är?}\\
Sammanhållningen var hög genomsnittet på de som svarade var 4,2.\\
\textbf{Varför tror du sammanhållningen är som den är?}\\
Kick-off och att arbeta tillsammans tyckte de svarande studenterna var viktiga faktorer till bra sammanhållning. De som kunde försämra sammanhållningen var att man upplevde att de andra i gruppen inte gjorde sin del eller att projektet i allmänhet inte gick så bra.\\\\För mer detaljerade resultat se bilaga [BIFOGA SVAR].

\section{Diskussion}
\label{sec:discussion-hampus}

%% Skriv här

\section{Slutsatser}
\label{sec:conclusions-hampus}

%% Skriv här

%%%%%%%%%%%%%%%%%%%%%%%%%%%%%%%%%%%%%%%%%%%%%%%%%%%%%%%%%%%%%%%%%%%%%%
