\chapter{Hur påverkas ett team av sin arbetsmiljö?}
\chapterprecis{\LARGE{---- Hampus Dunström ----}}
\label{cha:indiv-report-hampus}

\section{Inledning}
\label{sec:introduction-hampus}

Examen kommer bara närmare och närmare. Därav har jag börjat fundera på vad jag vill ha utav en arbetsplats. Det jag kommit fram till är en bra arbetsmiljö där jag trivs. Det tror jag är en nyckel till framgång för när man trivs presterar och utvecklas man som mest. 

\subsection{Syfte}
\label{sec:purpose-hampus}

Att försöka skapa lite mer klarhet hur  arbetsmiljön påverkar mjukvaruutveckling i ett team. Genom att öka förståelsen av orsaker och konsekvenser av val inom arbetsmiljön för teamet så kan bättre val göras i framtiden.

\subsection{Frågeställningar}
\label{sec:issue-hampus}

\begin{enumerate}
\item Hur ser arbetsmiljön ut för teamen i kursen \textit{Kandidatprojekt i programvaruutveckling}?
\item Hur påverkar arbetsmiljön sammanhållningen i teamet?
\item Hur påverkas gruppen av att få tillgång till ett eget rum där kandidatarbetet kan utföras jämfört med grupper som inte fått det?
\end{enumerate}

\subsection{Definitioner, akronym och förkortningar}
Följande definitioner och förkortningar används på flera ställen i denna del av rapporten:

\begin{itemize}
\item Småföretag - Företag med färre än 50 anställda.
\item Stora företag - Företag med fler än 50 anställda.
\item Allmänbelysning - Medelbelysningsstyrkan mätt i horisontalplanet 85 cm över golvet.
\item lux - SI-enhet för illuminans också kallad belysningsstyrka.
\end{itemize}

\section{Bakgrund}
\label{sec:background-hampus}

Under studietiden har det genomförts en del projekt och grupparbeten alla med varierande resultat. Alla dessa projekt har haft olika förutsättningar i form av arbetsmiljö, vissa har utförts på olika platser varje gång man träffats andra på samma ställe varje gång. I vissa fall kan man lämna saker på sin arbetsplats ibland inte. Vissa grupparbeten har skett i högljudda miljöer andra i tysta. Vad har detta för betydelse för arbetets resultat och effektivitet? Denna undersökning görs i ett försök att ta reda på hur den optimala arbetsmiljön ser ut. 

\section{Teori}
\label{sec:theory-hampus}
Nedan följer information kring arbetsmiljö i småföretag och kontorsmiljö. Informationen kommer från Arbetsmiljöverkets hemsida och en rapport \textit{Arbetsmiljö- och hälsoarbete i småföretag} publicerad av Arbetslivsinstitutet.\cite{AV}\cite{smaforetag}

\subsection{Belysning}
Vid kontorsarbete och framförallt bildskärmsarbete är bra belysning viktigt. Arbetsmiljöverket rekommenderar att en arbetsplats är belyst med 500 lux och allmänbelysningen är 300 lux. När man mäter belysningen är det mängden ljus som träffar ytan som mäts. Belysningen är inte det enda man ska tänka på när det kommer till belysning. Reflektioner, skuggor och kontraster kan också vara dåligt för arbetsmiljön. Arbetsmiljöverket tar upp arbetsplatsplacering i förhållande till fönster när man utför bildskärmsarbete. De föreslår att man placerar skärmarna vinkelrätt mot fönstren för att minimera reflektioner. \cite{AVLjus}\cite{AVDator}

Konsekvenserna av dålig belysning är koncentrationssvårigheter, huvudvärk och ansträngningsskador på ögonen. De som arbetar mer än en timme i veckan med bildskärmsarbete bör tillgodoses med regelbundna synundersökningar i intervall på två till fem år beroende på ålder. Där yngre människor inte behöver det lika mycket som äldre. Arbetsgivaren ska dessutom införskaffa glasögon vid behov då detta är ett mycket viktigt arbetsredskap.\cite{AVLjus}

\subsection{Ljud}
Enligt arbetsmiljöverket så är det viktigt med en tyst miljö när man ska utföra arbete som kräver koncentration. Det är viktigt att avskärma störande ljudkällor såsom skrivare och ventilation. I kontorsmiljö är andra människor en stor källa till störande ljud inte på grund av ljudnivån utan eftersom man distraheras av vad som sägs. Då är det viktigt med respekt eller avskilda rum där man kan prata ostört och utan att störa. \cite{AVLjud}

\subsection{Kontor}
Det finns flera typer av kontor. Cellkontor, öppna kontor, kombikontor och flexkontor är några exempel. På ett cellkontor har alla arbetare ett eget rum eller bås. Det ger avskildhet, minimerar störande ljud och underlättar koncentration. Informationsflödena blir dock sämre och det kan bli svårt att överblicka lokalen.

Öppna kontor är kontor där man arbetar i stora gemensamma rum. Här är det lättare att arbeta i grupper och flexibiliteten är större. Större blir också risken för störande ljud och det kan vara svårt och koncentrera sig, speciellt för arbetstagare med vissa kognitiva svårigheter.

En kombination av cellkontor och öppna kontor kallas kombikontor. Här finns både öppna utrymmen och avskilda kontor.

I de tidigare kontorstyperna har oftast arbetstagarna sin egen plats. Så är det inte i flexkontor också kallade aktivitetsbaserade kontor. Här har inte arbetstagarna en fast plats utan man väljer en plats som passar arbetet man ska utföra. Det är i dessa flexkontor eller cellkontor som medarbetarna mår bäst enligt en forskningsstudie men det kan finnas fler orsaker än bara kontorstypen. \cite{AVKontor}

\subsection{Skillnader mellan små och stora företag}
Arbetsmiljö är väldigt viktigt och något som lätt försummas speciellt i mindre företag. Arbetsolyckor och arbetsskador vanligare i småföretag än större men sjukfrånvaron är högre i större företag än i småföretagen. Detta beror bland annat på att småföretag är beroende av enskilda anställda i större utsträckning än stora företag.\cite{smaforetag}


\section{Metod}
\label{sec:method-hampus}

För att besvara frågeställningarna gjordes en empirisk studie i form av en enkät utskickad till de andra deltagare i kursen \textit{Kandidatprojekt i programvaruutveckling}. Enkäten skickades ut som ett Google Formulär\cite{GForms} med frågorna:
\begin{itemize}
\item Beskriv kort projektet du arbetar med.*
\item Hur intresserad av projektet är du?*
\item Hur många gånger i veckan träffas hela gruppen?*
\item Hur stor del av ditt arbete gör du ensam?
\item Föredrar du att jobba ensam eller tillsammans med någon?*
\item Vart brukar du utföra ditt arbete, hemma, i skolan eller...?*
\item Hur är ljudvolymen på dessa platser?
\item Hur är belysningen på dessa platser?, trivs du med den?
\item Är det folk runt omkring dig där, isåfall vad gör dom?
\item Har ni fått tillgång till någon lokal, isåfall hur ofta används den?*
\item Hur tycker du sammanhållningen i ditt team är?*
\item Varför tror du sammanhållningen är som den är?
\end{itemize}
Frågorna med '*' på slutet var obligatoriska. Svaren på enkäten kan ses i bilaga \ref{svar_enkat_arbetsmiljo}. Utifrån denna data och de personliga erfarenheterna som samlats under utförandet av Kandidatprojektet gjordes sedan en diskussion och slutsatser drogs. Dessa finns beskrivna under Diskussion \ref{sec:discussion-hampus} och Slutsatser \ref{sec:conclusions-hampus}. 

\section{Resultat}
\label{sec:results-hampus}

Det var 27 av de 98 (27,5\%) studenter som läser kursen \textit{Kandidatprojekt i programvaruutveckling} som svarade på enkäten. Sammanfattade svar kan ses nedan:\\
\\\textbf{Hur intresserad av projektet är du?}\\ 
Genomsnittsintresset låg på 3.8 på en skala 1-5.\\\\
\textbf{Hur många gånger i veckan träffas hela gruppen?}\\
Grupperna träffades i genomsnitt 3.5 gånger per vecka.\\\\
\textbf{Hur stor del av ditt arbete gör du ensam?}\\
Större delen av arbetet gjordes i grupp, 16 studenter svarade 25\%-50\% 9 svarade  mindre än 25\% och 2 studenter svarade att de gjorde en större del än så ensamma.\\\\
\textbf{Föredrar du att jobba ensam eller tillsammans med någon?}\\
Det allra flesta studenterna föredrog att arbeta tillsammans med någon.\\\\
\textbf{Vart brukar du utföra ditt arbete, hemma, i skolan eller...?}\\
Skolan var den plats som de flesta gjorde större delen av sitt arbete. \\\\
\textbf{Hur är ljudvolymen på dessa platser?}\\
Större delen av de svarande tyckte ljudvolymen var låg eller bra.\\\\
\textbf{Hur är belysningen på dessa platser?, trivs du med den?}\\
Studenterna var nöjda med belysningen, flera uttryckte att de föredrog fönster med naturligt ljus och enstaka studenter uttryckte specifikt misstycke för lysrör.\\\\
\textbf{Är det folk runt omkring dig där, isåfall vad gör dom?}\\
Runt omkring arbetsplatsen svarade större delen att det var andra studenter som också arbetade med sina studier.\\\\
\textbf{Har ni fått tillgång till någon lokal, isåfall hur ofta används den?}\\
De flesta hade inte fått tillgång till någon specifik lokal men de som hade det använde den flitigt. Av de som inte fått någon lokal uttryckte flera att de bokade olika lokaler i skolan nästan varje dag.\\\\
\textbf{Hur tycker du sammanhållningen i ditt team är?}\\
Sammanhållningen var hög, genomsnittet på de som svarade var 4,2.\\\\
\textbf{Varför tror du sammanhållningen är som den är?}\\
Kick-off och att arbeta tillsammans tyckte de svarande studenterna var viktiga faktorer till bra sammanhållning. De som kunde försämra sammanhållningen var att man upplevde att de andra i gruppen inte gjorde sin del eller att projektet i allmänhet inte gick så bra.\\\\
För mer detaljerade resultat se bilaga \ref{svar_enkat_arbetsmiljo}.

\section{Diskussion}
\label{sec:discussion-hampus}
Diskussionen är uppdelade utifrån frågeställningarna nedan.

\subsection{Metod}
Att använda sig av en enkät var ett bra sätt och få in en större mängd data. Det skulle inte ha varigt rimligt att hålla ca 30 intervjuer. Däremot hade ett par kompletterande intervjuer kunnat vara givande i tolkningen av svaren. Enkäten skulle också ha kunnat förbättras. Det hade varit en fördel om enkäten hade med en fråga om vilken grupp studenten var i. Då skulle jag kunnat se bättre hur sammanhållningen var i de olika grupperna. En fråga om hur studenterna tyckte deras sammanhållning påverkades av arbetsmiljön hade också gett användbar information.	Men jag tror att det var bra och hålla enkäten kort för lång enkät och den hade inte fått så många svar. 

Enkäten skickades ut i ett tidigt skede. Flera andra enkäter skickades ut långt senare. Det hade nog blivit fler svar om en påminnelse hade skickats ut i samband med många av de andra enkäterna. Ett problem där var dock att jag ville kunna börja arbeta med min rapport tidigt. Det hade dock gått och börja på rapporten och sen göra ändringar ifall att påminnelsen gav nya svar med data som visade på nya saker eller motbevisade saker jag hittills skrivit. 

\subsection{Hur ser arbetsmiljön ut för teamen i kursen \textit{Kandidatprojekt i programvaruutveckling}?}
Majoriteten av arbetet i kursen gör studenterna i skolan. Där trivs studenterna även om lysrören kanske inte alltid ger optimalt arbetsljus och ibland kan lysrören låta irriterande. Ljudvolymen är de flesta också nöjda med även om det ibland på de öppna ytorna kan bli lite störande. Lösningen för många är hörlurar. Att lyssna på musik stänger ute distraherande samtal och underlättar för koncentrationen. Men när studenter bokar egna rum så är sällan ljudvolymen ett problem. När studenterna inte arbetar i skolan gör de det oftast hemma. Där svarade studenterna att de hade skrivbordslampor som gjorde ett bra jobb och de största problemen var störande grannar. Jag håller med svaren till enkäten. Vi har en bra arbetsmiljö här på skolan det enda problemet är att det kan vara svårt och få tag på salar. De studenter som lyfte dessa problem löste dem med att ta sig till Creactive eller jobba hemifrån och kommunicera över Slack. Vilket har fungerat men lite fler grupprum hade varit uppskattat. 

\subsection{Hur påverkar arbetsmiljön sammanhållningen i teamet?}
Det var väldigt få som inte tyckte att det var en bra sammanhållning i sitt team. Bara en person tyckte att sammanhållningen var dålig (sämre än 3 på en skala 1-5) i sitt team. Detta gjorde att det inte går att hitta ett bra samband. Istället så såg jag att studenternas intresse för projekten också var hög. Detta fick mig att fundera på huruvida arbetsmiljön kanske inte riktigt hunnit sjunka in. Att i korta projekt så är arbetsmiljön kanske inte så viktig så länge man får göra något intressant och roligt.

För att få en bättre inblick i hur arbetsmiljön borde en undersökning göras där man ger några grupper en bra arbetsmiljö, några en dålig och en kontrollgrupp med en vanlig arbetsmiljö. Projekten grupperna ska arbeta med skulle sen lottas ut och efter de projekten genomförts skulle bättre resultat kunna uppnås. 

En högre deltagarsiffra än 27,5\% i undersökningen skulle också vara bra. Det kan vara så att de som haft dålig sammanhållning i gruppen har haft dålig motivation och därför inte varit lika aktiva med att svara på olika enkäter. Då svara på enkäter inte är den roligaste sysselsättningen.

\subsection{Hur påverkas gruppen av att få tillgång till ett eget rum där kandidatarbetet kan utföras jämfört med grupper som inte fått det?}
En av de första tydliga skillnaderna mellan olika grupper är att vissa fick tillgång till egna rum på universitetet. Rum som endast den gruppen använde. Detta resulterade i att dessa grupper utförde mycket av sitt arbete i öppna kontorsutrymmen. Andra grupper som inte blev tilldelade rum att arbeta i har arbetat i något som mer kan liknas vid aktivitetsbaserade kontor. Då de bokat salar i skolan, arbetat på Creactive i Mjärdevi eller hemma. Creactive i Mjärdevi kan ses som ett stort öppet kontorslandskap med arbetsplatser och ytor avsedda för umgänge och samtal. Våran projektgrupp fick ett rum på universitetet då vårt projekt till en början var knuten till en viss hårdvara som bara fanns där. Det var en stor fördel i början av projektet när mycket samarbete behövdes. Men lite längre fram i projektet uppstod en del störningsmoment på grund av att vi var alldeles för många på en för liten yta. Att inte få ett rum kan alltså ha varit en fördel. Arbetsmiljöverket skriver till exempel om att det finns forskning som visar att aktivitetsbaserade kontor är bättre för medarbetarnas välmående än öppna kontorslösningar. 

Men detta kan också bero på helt andra saker. Vårt projekt fick en hel del problem vilket gjorde att moralen i gruppen blev sämre. Det fanns till exempel en annan grupp som fick ett liknande rum vars projekt inte stötte på sådana problem och där har moralen varit högre och sammanhållningen bättre. Det är precis som Arbetsmiljöverket även skriver att välmåendet inte bara behöver bero på utformningen av kontoret utan kan bero på många faktorer. Min undersökning visar bland annat att över lag är sammanhållningen hög precis som intresset för de olika projektet över lag är hög. Detta i kombination med teambuilding som lyfts fram i undersökningen kan vara några av de andra faktorer som påverkar sammanhållningen. Det var dessutom bara 27,5\% av studenterna som läser kursen som svarade på enkäten. Så undersökningen kan ha missat en hel del.

\section{Slutsatser}
\label{sec:conclusions-hampus}

Mina slutsatser utifrån frågeställningar följer nedan.

\subsection{Hur ser arbetsmiljön ut för teamen i kursen \textit{Kandidatprojekt i programvaruutveckling}?}
Teamen i kursen har en bra arbetsmiljö både i skolan och hemma utifrån min undersökning. De öppna ytorna i skolan då studenter inte bokar en sal kan ha en störande ljudvolym och ibland saknar de fönster istället för lysrör. 

\subsection{Hur påverkar arbetsmiljön sammanhållningen i teamet?}
Det gick inte att dra en direkt relation mellan arbetsmiljön och sammanhållningen i teamen. Det var alldeles för många faktorer och mycket mer efterforskningar krävs. Att det dessutom bara var 27,5\% av de som läste kursen som svarade på enkäten gör att undersökningen skulle behövt mer data för att kunna ta mer konkreta slutsatser.

\subsection{Hur påverkas gruppen av att få tillgång till ett eget rum där kandidatarbetet kan utföras jämfört med grupper som inte fått det?}
Grupperna som blivit tilldelade rum har enligt undersökningen använt dem flitigt och varierat sin arbetsmiljö mindre än de som inte fått något rum. De som inte fått tillgång till något rum har varit mer flexibla och valt att boka salar, jobba hemifrån eller i andra lokaler beroende på vad som passat bäst. 

De grupper som flitigt använd sina rum har dock haft det lättare med planering och kommunikation. Eftersom de flesta var på plats i rummet under arbetstid så kunde man i dessa grupper lätt få till ett samtal öga mot öga. I början av arbetet under förstudie och i början var detta till stor hjälp och gjorde att grupper med rum kom igång med sitt arbete fortare än andra grupper. 


%%%%%%%%%%%%%%%%%%%%%%%%%%%%%%%%%%%%%%%%%%%%%%%%%%%%%%%%%%%%%%%%%%%%%%
