\chapter{Hur påverkas ett team av sin arbetsmiljö?}
\label{cha:indiv-report-person}

Det här appendixet innehåller följande sektioner.

\section{Inledning}
\label{sec:introduction-person}

%% Skriv här

\subsection{Syfte}
\label{sec:purpose-person}

Att försöka skapa lite mer klarhet hur  arbetsmiljön påverkar mjukvaruutveckling i ett team. Genom att öka förståelsen av orsaker och konsekvenser av val inom arbetsmiljön för teamet så kan bättre val göras i framtiden.

\subsection{Frågeställning}
\label{sec:issue-person}

Hur påverkar arbetsmiljön kring ett team dess arbetssätt?

Hur påverkar arbetsmiljön sammanhållningen i teamet?

Hur påverkas gruppen av att få tillgång till ett eget rum där kandidatarbetet kan utföras jämfört med grupper som inte fått det?

\section{Bakgrund}
\label{sec:background-person}

Under studietiden har det genomförts en del projekt och grupparbeten alla med varierande resultat. Alla dessa projekt har haft olika förutsättningar i form av arbetsmiljö, vissa har utförts på olika platser varje gång man träffats andra på samma ställe varje gång. I vissa fall kan man lämna saker på sin arbetsplats ibland inte. Vissa grupparbeten har skett i högljudda miljöer andra i tysta. Vad har detta för betydelse för arbetets resultat och effektivitet? Denna undersökning görs i ett försök att ta reda på hur den optimala arbetsmiljön ser ut. 

\section{Teori}
\label{sec:theory-person}

%% Skriv här

\section{Metod}
\label{sec:method-person}

Det gjordes en empirisk studie i form av enkäter och intervjuer med andra deltagare i kursen Kandidatprojekt i programvaruutveckling. Enkäten skickades ut som ett Google Formulär [BIFOGA FORMULÄRET OCH HÄNVISA!], intervju frågorna och svaren kan ses i bilaga [BIFOGA FORMULÄRFRÅGOR OCH SVAR!]. Det gjordes också litteraturstudier av liknande undersökningar och artiklar i ämnet som kunde hittas på internet. Utifrån denna data och de personliga erfarenheterna som samlats under utförandet av Kandidatprojektet gjordes sedan en diskussion och slutsatser drogs. Dessa finns beskrivna under Diskussion och Slutsatser. [LÄGG TILL HÄNVISNINGAR]

Valet av både enkät och intervjuer gjordes för att både få kvantitativ data från enkäten och lite mer kvalitativ data från intervjuerna. Detta tillsammans med resultaten från litteraturstudien gav en bra bredd av datainsamling.

\section{Resultat}
\label{sec:results-person}

%% Skriv här

\section{Diskussion}
\label{sec:discussion-person}

%% Skriv här

\section{Slutsatser}
\label{sec:conclusions-person}

%% Skriv här

%%%%%%%%%%%%%%%%%%%%%%%%%%%%%%%%%%%%%%%%%%%%%%%%%%%%%%%%%%%%%%%%%%%%%%
