\chapter{Teori}
\label{cha:theory}

\section{Punktmoln}
Ett punktmoln är en mängd av punkter i ett tredimensionellt koordinatsystem. Varje punkt i punktmolnet består av tre värden som representerar punktens XYZ-koordinater. Tillsammans kan dessa punkter representera ett objekt. Detta görs ofta genom att punkterna representerar utkanterna av objektet. \cite{point_cloud}. 

Det finns två typer av punktmoln, ett så kallat komplett punktmoln och ett icke komplett punktmoln. Ett komplett punktmoln, se figur \ref{fig:point_cloud_torus}, betyder att punktmolnet innehåller samtliga punkter som behövs för att representera objektet. Ett icke komplett punktmoln, se figur \ref{fig:point_cloud_church}, betyder att punktmolnet endast representerar en del av objektet. Flera stycken icke kompletta punktmoln kan kombineras för att skapa ett komplett punktmoln. Denna process kallas registrering och kommer förklaras noggrannare senare i rapporten.

\begin{figure}[H]
	\centering
	\includegraphics[width=30mm]{figures/Point_cloud_torus1.png}
	\includegraphics[width=30mm]{figures/Point_cloud_torus2.png}
	\includegraphics[width=30mm]{figures/Point_cloud_torus3.png}
	\includegraphics[width=30mm]{figures/Point_cloud_torus4.png}
	\caption{Ett komplett punktmoln som representerar ett torusobjekt.}
	\label{fig:point_cloud_torus}
\end{figure}

\begin{figure}[H]
	\centering
	\includegraphics[width=100mm]{figures/icke_komplett_moln_kyrka.png}
	\caption{Ett icke komplett punktmoln som representerar en del av en kyrka.}
	\label{fig:point_cloud_church}
\end{figure}

\section{Point Cloud Library}

Point Cloud Library (PCL) är ett bibliotek till C++ som bidrar med algoritmer till att behandla 3D-objekt. PCL innehåller bland annat algoritmer för:

\begin{itemize}
	\item filtrering
	\item registrering
	\item ytrekonstruering.
\end{itemize}
PCL är alltså ett bibliotek avsett att ge stöd till att behandla punktmolnsdata, i form av olika bearbetningsalgoritmer \cite{rusu20113d}.

PCL använder sig av ett eget visualiseringsbibliotek för att visualisera punktmoln. Visualiseringsbiblioteket i PCL integrerar med visualiseringsverktyget VTK \cite{VTK_book}\cite{rusu20113d}.

Nedan beskrivs de olika algoritmerna från PCL som är relevanta för projektet.

\subsection{Filtrering}
PCLs filtreringsbibliotek innehåller algoritmer för att filtrera bort skräppunkter från ett punkt\-moln. Dessa skräppunkter kan uppstå vid skanning av ett objekt eftersom avståndskameran kan reagera på bakgrunden och inte bara på objektet. Detta gör att skräppunkter infaller i punkmolnet. För att kunna behandla punktmolnet måste dessa skräppunkter först filtreras bort.

PCL har stöd för flera olika metoder för filtrering. Ett sätt att filtrera punktmoln är att ta bort alla punkter som ligger utanför ett visst intervall av koordinater. Syftet med detta är att ta bort sådan som måste ligga utanför det skannade objektet. En annan metod är att för varje punkt i punktmolnet kontrollera hur många grannar den punkten har inom en viss radie. Alla punkter som har under ett visst antal grannar tas bort vilket leder till att enstaka punkter som inte ligger i närheten av något objekt tas bort. Ofta används dessa filtreringsmetoder tillsammans för att försäkra sig om att all skräpdata tagits bort \cite{pcl_filtering}.

För att undvika att punktmoln blir för stora, det vill säga att de får för många punkter vilket gör att att de tar för lång tid att behandla, kan nedsampling användas för att ta bort punkter och samtidigt uppskatta den yta som representeras så nära som möjligt. För att göra detta i PCL delas punktmolnet som ska nedsamplas in i voxels, vilket kan ses som att punktmolnet delas in i ett antal kuber. Sedan ersätts alla punkter i varje voxel med en punkt som ligger i centrum för alla punkterna i voxeln. På detta sätt ges en nära uppskattning av den representerade ytan med mycket färre punkter än det ursprungliga punktmolnet \cite{voxel_grid}.
 

\subsection{Registrering}
\label{sec:theory_registration}
Att kombinera två punktmoln till ett punktmoln som innehåller samma information kallas för att registrera punktmoln. Detta är en komplicerad process där både translationen och rotationen, runt alla axlar, kan behöva ändras. Målet med parvis registrering är att flytta ett punktmoln till det andra punktmolnet så att alla punkter i de två separata punktmolnen överlappar med varandra. I PCL finns det ett registreringsbibliotek som innehåller algoritmer för registrering \cite{pcl_registration}.

För att registrera punktmoln finns ett antal olika algoritmer med olika för- och nackdelar. En vanlig algoritm är \textit{Iterative Closest Point} (ICP). Det är en så kallad parvis registreringsalgoritm som registrerar enbart ett punktmoln till ett annat. ICP går ut på att minimera summan av avståndet från varje punkt i det ena punktmolnet till varje punkt i det andra punktmolnet.

För att begränsa antalet punkter som måste beräknas används parametern \textit{Max correspondence distance} så att enbart de punkter inom avståndet räknas in i summan för varje punkt. \textit{Transformation epsilon} är ett mått på hur mycket transformationen för punktmolnet som ska matchas in får ändras för varje iteration. Genom att öka detta epsilon kan, vid behov, algoritmen flytta punktmolnet längre och därmed nå sitt mål på färre iterationer. Nackdelen med ett för högt epsilon är dock att punktmolnet kan flyttas förbi optimum som därmed slösar tid. Slutligen kan \textit{Max iterations} användas för att avgöra hur många gånger algoritmen ska köras för varje parvis registrering. Genom att öka antalet iterationer ökar tidsåtgången men resultatet förbättras också \cite{pcl_icp_docs}.

\subsection{Ytrekonstruering}
För att återskapa ytor från en 3D-modell används PCLs ytrekonstrueringsbibliotek. I projektets sammanhang betyder det att ett punktmoln används för att skapa en vattentät modell. De algoritmer som biblioteket tillhandahåller kan användas för att generera en mesh utifrån ett komplett punktmoln \cite{pcl_surface_reconstruction}.  

Meshning av ett punktmoln görs efter att punktmolnen registrerats så de bildar ett komplett punktmoln. Att skapa en 3D-mesh av ett komplett punktmoln innebär att skapa en yta utan punkter. Punktmolnets 3D-mesh är alltså en vattentät 3D-modell uppbyggd av polygoner. En algoritm som kan användas för att mesha ett punktmoln är en trianguleringsalgoritm vid namn \textit{Poisson surface reconstruction}. Denna algoritm körs på ett punktmoln med normaler för att erhålla en yta baserad på grannarna till respektive punkt. För detta krävs alltså ett punktmolnsobjekt som innehåller normalerna till varje punkt i punktmolnet. I detta punktmoln med normaler kan sedan önskade parametrar sättas innan själva trianguleringsalgoritmen körs för att generera den önskade meshen \cite{pcl_surface_reconstruction}\cite{pcl_triangulation_algorithm}. 


\section{3D-skrivare}
En 3D-skrivare kan användas för att skriva ut 3D-objekt i olika material vanligast är plast. Detta görs utifrån datormodeller som skapats i till exempel CAD eller med system som 3DCopy.. För att kunna göra detta har skrivaren ett styrkort som exekverar så kallad G-code \cite{gcode}. Denna G-code beskriver vilka motorer som ska flyttas och hur mycket. En 3D-modell som är sparad i datorn, i form av en mesh, måste alltså behandlas genom att använda en algoritm för att generera G-code utifrån modellen. Detta görs ofta genom en typ av programvara som kallas för slicer. Namnet kommer av att programmet delar upp modellen i olika lager och sedan genererar hur skrivaren måste röra sig för att skriva ut ett lager. På detta sätt byggs sedan hela den ursprungliga modellen upp lager för lager. Två vanliga slicer-programvaror är \textit{Cura} \cite{cura} och \textit{Slic3r} \cite{slic3r}.

\section{Systemanatomi}
En systemanatomi är en beskrivning av ett systems funktionalitet. Ofta består en systemanatomi av en graf, där noder representerar systemets olika funktioner och bågar representerar beroenden. Systemanatomin säger ingenting om hur systemet ska implementeras utan ger en utvecklingsgrupp en bra översikt av vad systemet ska göra. En systemanatomi kan inte ersätta andra modeller men är ett bra komplement \cite{system_anatomy}.


\section{Hållbar utveckling}

Hållbar utveckling är en relevant aspekt i dagens utveckling av programvaror eftersom de används kontinuerligt av samhället. En programvara som utvecklats kan både hjälpa samhället framåt men också påverka samhället negativt, beroende på hur den producerats och hur den används \cite{raturi2014developing}.

Hållbarhet definieras som "förmåga att uthärda" och "bevara funktionen hos ett system under en utsträckt tidsperiod". Att analysera hållbarhet hos ett mjukvarusystem innebär för de som utvecklar systemet att ha dessa fyra områden i åtanke \cite{lago2015framing}:

\begin{itemize}
	\item Ekonomiskt -  Systemet ska bevara kapital och värde.
	\item Socialt - Systemet ska underhålla samhället.
	\item Miljö - Systemet ska skydda den mänskliga välfärden genom att skydda naturens tillgångar.
	\item Tekniskt - Systemet ska utvecklas för att stödja långtidsanvändning.
\end{itemize}
En mer ingående beskrivning om hur 3DCopy främjar hållbar utveckling diskuteras i avsnitt \ref{disc:hållbar_utveckling}.


%%%%%%%%%%%%%%%%%%%%%%%%%%%%%%%%%%%%%%%%%%%%%%%%%%%%%%%%%%%%%%%%%%%%%%
%%% theory.tex ends here
