\chapter{Teori}
\label{cha:theory}


%% Vad behöver läsaren veta för att förstå resten av rapporten? %%

The main purpose of this chapter is to make it obvious for
the reader that the report authors have made an effort to read
up on related research and other information of relevance for
the research questions. It is a question of trust. Can I as a
reader rely on what the authors are saying? If it is obvious
that the authors know the topic area well and clearly present
their lessons learned, it raises the perceived quality of the
entire report.

After having read the theory chapter it shall be obvious for
the reader that the research questions are both well
formulated and relevant.

The chapter must contain theory of use for the intended
study, both in terms of technique and method. If a final thesis
project is about the development of a new search engine for
a certain application domain, the theory must bring up related
work on search algorithms and related techniques, but also
methods for evaluating search engines, including
performance measures such as precision, accuracy and
recall.

The chapter shall be structured thematically, not per author.
A good approach to making a review of scientific literature
is to use \emph{Google Scholar} (which also has the useful function
\emph{Cite}). By iterating between searching for articles and reading
abstracts to find new terms to guide further searches, it is
fairly straight forward to locate good and relevant
information, such as \cite{test}.

Having found a relevant article one can use the function for
viewing other articles that have cited this particular article,
and also go through the article’s own reference list. Among
these articles on can often find other interesting articles and
thus proceed further.

It can also be a good idea to consider which sources seem
most relevant for the problem area at hand. Are there any
special conference or journal that often occurs one can search
in more detail in lists of published articles from these venues
in particular. One can also search for the web sites of
important authors and investigate what they have published
in general.

This chapter is called either \emph{Theory, Related Work}, or
\emph{Related Research}. Check with your supervisor.



\section{Hållbar utveckling}
Hållbar utveckling är en relevant aspekt i dagens utveckling av programvaror eftersom de används kontinuerligt av samhället. Effekterna som en programvara bidrar med i samhället beror på hur den har producerats och hur den används av konsumenterna. Detta gör att effekterna av som en programvara bidrar med i samhället kan vara både positiva men också djupt negativa \cite{raturi2014developing}. Hållbarhet definieras som \textit{förmåga att uthärda} och \textit{bevara funktionen hos ett system under en utsträckt tidsperiod}. Att analysera hållbarhet hos ett mjukvarusystem innebär för de som utvecklar systemet att väga in dessa fyra områden i åtanke \cite{lago2015framing}:

\begin{itemize}
	\item Ekonomiskt -  Systemet ska bevara kapital och värde.
	\item Socialt - Systemet ska underhålla samhället.
	\item Miljö - Systemet ska skydda mänskliga välförden genom att skydda naturens tillgångar.
	\item Tekniskt - Systemet ska utvecklas för att stödja långtidsanvändning.
\end{itemize}

Ett system för 3D-kopiering har många positiva effekter på samhället. Systemet gynnar framförallt miljön eftersom att tillverka en önskad produkt istället för att köpa en fabrikstillverkad sparar avsevärt på naturens tillgångar. Dels kommer 3D-kopieringssystemet att att använda mindre energi och dessutom kommer transporterna till och från affären att minska, vilket leder till mindre koldioxidutsläpp. Kreiger \cite{kreiger2013environmental} har gjort en studie där han skriver ut 3D-produkter i plast och jämför kostnaden för att tillverka produkter av plast i en fabrik. Med kostnaden menas den energi som går åt från råmaterial till färdig produkt samt kostnaden som går åt för transport. Det visar sig att tillverka en produkt i en 3D-skrivare kräver mellan 41 till 64 procent mindre energi än att fabrikstillverka produkten. Förklaringen till detta är att produkter som skrivs ut i en 3D-skrivare kan göras mer ihåliga och således kräver de också mindre material. 

Det finns dessutom relaterade studier som visar att det blir billigare att skriva ut en produkt i en 3D-skrivare istället för att köpa en fabrikstillverkad \cite{wittbrodt2013life}. Det här främjar samhället i positiv beaktning eftersom det helt enkelt blir billigare för konsumenter att införskaffa sig de produkter de önskar. 

\subsection{Förbättringspunkter till vårat system}

Hur skulle vi kunna göra annorlunda i kravprocessen? 

Vad har vi tagit hänsyn till och vad skulle vi kunna gjort annorlunda?

Hur kan vi bedöma de krav vi satt på systemet?

...

Eventuella förbättringar gällande vår kravhantering:
\begin{itemize}
	\item Sätta upp icke funktionella krav som främjar hållbar utveckling
	\item Systemet ska klara av att producera en komplett produkt på så få skanningar som möjligt. Det är slöseri med energi att skanna för många gånger
	\item optimera tiden det tar att göra en skanning. Detta är också bra för energisynpunkt (dumt att systemet står på i onödan eftersom det tar en stund att skriva ut någonting).
	\item 
\end{itemize}


%%%%%%%%%%%%%%%%%%%%%%%%%%%%%%%%%%%%%%%%%%%%%%%%%%%%%%%%%%%%%%%%%%%%%%
%%% theory.tex ends here
