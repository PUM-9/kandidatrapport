\chapter{Kvalitetsarbete i praktiken. Fredrik Wallström}
\label{cha:indiv-report-person}

\section{Inledning}
\label{sec:introduction-person}

De flesta företag ute på marknaden idag strävar mot att utveckla bättre produkter, vassare tjänster och smidigare processer. Dessa produkter, tjänster och processer blir alltmer komplexa, samtidigt som företagen strävar efter att bli alltmer kostnadseffektiva. I arbetet med att utveckla nya produkter spelar kvalitetsarbetet en betydande roll. Det är ett problem att effektivisera kvalitetsarbetet för att hinna med dagens utveckling, då inte ens konsumenterna själva hinner med. Kvalitetssäkring är generellt ett komplext ämne att behandla, på grund av dess diffusa definition men det står klart att det är en viktig komponent i utvecklingen av dagens produkter, tjänster och processer.

\subsection{Syfte}
\label{sec:purpose-person}

Syftet med denna rapport är att ta reda på hur man kan använda sig av kvalitetsarbete i praktiken vid utveckling av ett mjukvarusystem. Syftet är också att undersöka möjligheterna till att utveckla och effektivisera dagens kvalitetsarbete. Detta undersöks eftersom det blir alltmer aktuellt på grund av den snabba utvecklingen av produkter, tjänster och processer bland företagen i dagens samhälle.


\subsection{Frågeställning}
\label{sec:issue-person}

De frågeställningar som rapporten kommer ta upp är vad det egentligen innebär med att kvalitetssäkra mjukvaran för en produkt samt vilka metoder det finns för ändamålet.  Rapporten kommer även undersöka vad effekterna blir av att kvalitetssäkra mjukvaran med en specifik metod, nämligen kodgranskning. Slutligen kommer rapporten att behandla vad de egentliga kostnaderna är med att kvalitetssäkra en produkt med kodgranskning. Kommer effekterna för kvalitetssäkringen kompensera för den faktiska tidsåtgången?

\newpage
\subsection{Eventuella tänkta referenser}

Rapporten kommer att samla information utifrån artiklar som är publicerade på konferenser. Rapporten kommer också att utgå ifrån personliga erfarenheter från att arbeta som kvalitetssamordnare för ett projekt. Nedan listas de artiklar som eventuellt kommer att användas till rapporten, artiklarna listas nu som en länk, vilket självklart ändras vid färdigställning av rapporten.

\begin{itemize}
	\item \url{http://dl.acm.org.e.bibl.liu.se/ft_gateway.cfm?id=2597076&ftid=1467105&dwn=1&CFID=730226200&CFTOKEN=95764016}
	\item \url{https://www.researchgate.net/profile/Stefan_Wagner4/publication/221494816_Quality_models_in_practice_A_preliminary_analysis/links/0912f5072935824753000000.pdf}
	\item \url{http://dl.acm.org.e.bibl.liu.se/ft_gateway.cfm?id=1083296&ftid=328676&dwn=1&CFID=730226200&CFTOKEN=95764016}
	
\end{itemize}

\section{Bakgrund}
\label{sec:background-person}

%% Skriv här

\section{Teori}
\label{sec:theory-person}

%% Skriv här

\section{Metod}
\label{sec:method-person}

%% Skriv här

\section{Resultat}
\label{sec:results-person}

%% Skriv här

\section{Diskussion}
\label{sec:discussion-person}

%% Skriv här

\section{Slutsatser}
\label{sec:conclusions-person}

%% Skriv här

%%%%%%%%%%%%%%%%%%%%%%%%%%%%%%%%%%%%%%%%%%%%%%%%%%%%%%%%%%%%%%%%%%%%%%
%%% person-report.tex ends here
