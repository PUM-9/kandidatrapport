\chapter{Kvalitetsarbete i praktiken}
\label{cha:indiv-report-wallstrom}
\chapterprecis{\LARGE{---- Fredrik Wallström ----}}

\section{Inledning}
\label{sec:introduction-wallstrom}

De flesta företag ute på marknaden idag strävar mot att utveckla bättre produkter, vassare tjänster och smidigare processer. Dessa produkter, tjänster och processer blir alltmer komplexa, samtidigt som företagen strävar efter att bli alltmer kostnadseffektiva. I arbetet med att utveckla nya produkter spelar kvalitetsarbetet en betydande roll. Det är ett problem att effektivisera kvalitetsarbetet för att hinna med dagens utveckling, då inte ens konsumenterna själva hinner med. Kvalitetssäkring är generellt ett komplext ämne att behandla, på grund av dess diffusa definition men det står klart att det är en viktig komponent i utvecklingen av dagens produkter, tjänster och processer.

\subsection{Syfte}
\label{sec:purpose-wallstrom}

Syftet med denna rapport är att ta reda på vad det innebär att kvalitetssäkra ett mjukvarusystem och vilka metoder det finns för ändamålet. Syftet är också att se över hur kvalitetsarbetet används i praktiken, samt om det finns möjligheter att utveckla och effektivisera detta arbete. Detta undersöks eftersom det blir alltmer aktuellt på grund av den snabba utvecklingen av produkter, tjänster och processer bland företagen i dagens samhälle.

\subsection{Frågeställning}
\label{sec:issue-wallstrom}

De frågeställningar som denna rapport kommer ta upp är:

\begin{itemize}
	\item Vad är effekterna av att kvalitetssäkra mjukvaran för en produkt med den specifika metoden kodgranskning?
	\item Vad är de egentliga kostnaderna med att kvalitetssäkra en produkts mjukvara med kodgranskning, kommer effekterna för kvalitetssäkringen kompensera för den faktiska tidsåtgången?
\end{itemize}

\clearpage

\subsection{Eventuella tänkta referenser}

Rapporten kommer att samla information utifrån artiklar som är publicerade på konferenser. Rapporten kommer också att utgå ifrån personliga erfarenheter från att arbeta som kvalitetssamordnare för ett projekt. Nedan listas de artiklar som eventuellt kommer att användas till rapporten, artiklarna listas nu som en länk, vilket självklart ändras vid färdigställning av rapporten.

\begin{itemize}
	\item \url{http://dl.acm.org.e.bibl.liu.se/ft_gateway.cfm?id=2597076&ftid=1467105&dwn=1&CFID=730226200&CFTOKEN=95764016}
	\item \url{https://www.researchgate.net/profile/Stefan_Wagner4/publication/221494816_Quality_models_in_practice_A_preliminary_analysis/links/0912f5072935824753000000.pdf}
	\item \url{http://dl.acm.org.e.bibl.liu.se/ft_gateway.cfm?id=1083296&ftid=328676&dwn=1&CFID=730226200&CFTOKEN=95764016}
	\item 
	\url{	http://dl.acm.org.e.bibl.liu.se/ft_gateway.cfm?id=1050862&ftid=304061&dwn=1&CFID=730226200&CFTOKEN=95764016}

\end{itemize}

\section{Bakgrund}
\label{sec:background-wallstrom}
\subsection{Kvalitetssäkring}
Att säkerställa hög kvalité på ett system idag, innebär inte bara att testa och analysera systemet att det fungerar korrekt. För att försäkra sig själv och framförallt övertyga användaren om att systemet är i bra skick och presterar väl, kräver hög nogrannhet och disciplinerat kvalitetsarbete. Kvalitetssäkran är en process som genomförs kontinuerligt under uppbyggnaden av ett system och ingenting som appliceras vid systemet slutfas \cite{feldman2005quality}.

Att genomföra en kvalitetssäkrande process innebär enligt definiton, att tillhandahålla en försäkran om att programvaruprodukten och processer i produktens livscykel överensstämmer med deras specifika krav och håller fast vid de uppsatta planerna. \textit{Process}, är nyckelordet för kvalitetssäkran, det innebär alltså att kvalitetssäkran är en metod och inte en enstaka teknik. En annan viktig aspekt för kvalitetssäkran är att det inte behandlas som ett filosofiskt problem, utan mer som ett mätbart problem. Slutligen innebär kvalitetssäkran om att ge garantier och trovärdighet, produkten ska fungera rätt och folk ska tro att den fungerar rätt \cite{feldman2005quality}.

Kvalitetssäkran innefattar \textit{testning} som en av de viktigare aktiviteterna. Det finns dock ett ordspråk som säger att det inte går att testa kvaliten i produkten, utan att en testplan enbart kan fånga fel och ge ett mått på kvaliten. Det finns olika kvalitetssäkrande processer, de definieras för att passa behoven hos produkten. En process kan definieras för att försäkra sig om att designen är lämplig, implementationen är nogrann och att produkten möter alla krav innan publicering. Denna process kan även utvecklas till att analysera de upptäckta defekterna och sedan ständigt förbättra dessa \cite{feldman2005quality}.


\subsection{Kodgranskning}


\clearpage

\section{Teori}
\label{sec:theory-wallstrom}

Samhället idag ställer höga krav på tekniken. Eftersom dagens teknik i stor utsträckning består av produkter med ett integrerat system, innebär det att mjukvaruutvecklingen blir alltmer komplex. Mjukvaruutveckling består idag av en nogrann integrering av olika faktorer, som till exempel design, utveckling, användarvänlighet samt kompetenta programmerare med förmågan att sammarbeta. Det här innebär att idag läggs det alltmer vikt på att kvalitetsäkra ett mjukvarusystem för att tillfredsställa samhället. Oavsett hur avancerade verktyg, tekniker och metoder som har använts för att framställa en produkt är produkten oanvändbar om den inte gått igenom en kvalitetsäkrande process \cite{gill2005factors}.





\section{Metod}
\label{sec:method-wallstrom}

%% Skriv här

\section{Resultat}
\label{sec:results-wallstrom}

%% Skriv här

\section{Diskussion}
\label{sec:discussion-wallstrom}

%% Skriv här

\section{Slutsatser}
\label{sec:conclusions-wallstrom}

%% Skriv här

%%%%%%%%%%%%%%%%%%%%%%%%%%%%%%%%%%%%%%%%%%%%%%%%%%%%%%%%%%%%%%%%%%%%%%
%%% person-report.tex ends here
