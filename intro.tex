\chapter{Inledning}
\label{cha:introduction}

%% Vad är det för spännande projekt ni gjort?
%% Varför skall jag forsätta läsa rapporten?
\section{Motivering}
\label{sec:motivation}

Denna rapport redogör för hur ett system för 3D-kopiering har konstruerats. Tekniken för att kunna skriva ut 3D-objekt har funnits i många år men först på 2010-talet har 3D-skrivare kommit ner i pris så pass att vanliga konsumenter kan köpa en. Även i industrin har det blivit allt vanligare att använda 3D-skrivare för prototyper. Det finns dock ett problem: för att kunna använda 3D-skrivaren måste man antingen kunna CAD-mjukvara eller förlita sig på andra människors 3D-modeller. Genom att använda ett system för 3D-kopiering kan ett verkligt objekt istället kopieras.


%% Vad skall rapporten leda till? (Beskriva projekt, samla 
%% erfarenheter, speciella djupdykningar)
\section{Syfte}
\label{sec:aim}

Syftet med detta projekt var att konstruera en mjukvara för att styra ett system som kopierar tredimensionella objekt. Projektet bygger vidare på ett tidigare kandidatarbete som utfördes våren 2016. Resultatet av det tidigare projektet var en mjukvara som styr ett rotationsbord och en linjärenhet med avståndskamera och genererar ett punktmoln (se definitioner). Syftet var alltså att vidareutveckla resultatet från det tidigare projektet till ett system som kan kopiera ett tredimensionellt objekt. Detta görs genom att ett flertal skanningar tas med olika rotationer och lutningar för att sedan registrera de resulterande punktmolnen till ett komplett punktmoln för objektet. Detta kompletta punktmoln omvandlas sedan till en 3D-mesh (se definitioner) och sparas i ett format som kan användas för att skriva ut objektet på en 3D-skrivare. Systemet är modulärt, välbyggt och enkelt att förstå för att göra det möjligt för vidareutveckling, forskning samt för eventuell användning till laborationer.   


\section{Frågeställning}
\label{sec:research-questions}

Projektgruppen vill att systemet ska vara användbart för kunden. Systemet ska tänkas användas till vidareforskning eller för laborationer i kurser på Linköpings universitet. En av frågeställningarna som rapporten kommer ta upp är således hur systemet kan implementeras så att det maximerar värdet för kunden. En annan frågeställning som rapporten kommer att ta upp är hur systemet kan vidareutvecklas för framtida projekt. Denna vidareutveckling kommer baseras på de erfarenheter som projektet medför.


This is where the research questions are described.
Formulate these as explicit questions, terminated with a
question mark. A report will usually contain several different
research questions that are somehow thematically connected.
There are usually 2-4 questions in total.

Examples of common types of research questions (simplified
and generalized):

\begin{enumerate}
\item Hur kan system X implementeras så att man skapar värde för kunden?

\item Vilka erfarenheter kan dokumenteras från programvaruprojekt Y som kan vara intressanta för framtida projekt?

\item Vilket stöd kan man få genom att skapa och följa upp en systemanatomi?

\item Specifika frågeställningar.

\end{enumerate}


Observe that a very specific research question almost always
leads to a better thesis report than a general research question
(it is simply much more difficult to make something good
from a general research question.)

The best way to achieve a really good and specific research
question is to conduct a thorough literature review and get
familiarized with related research and practice. This leads to
ideas and terminology which allows one to express oneself
with precision and also have something valuable to say in the
discussion chapter. And once a detailed research question
has been specified, it is much easier to establish a suitable
method and thus carry out the actual thesis work much faster
than when starting with a fairly general research question. In
the end, it usually pays off to spend some extra time in the
beginning working on the literature review. The thesis
supervisor can be of assistance in deciding when the research
question is sufficiently specific and well-grounded in related
research.

\section{Avgränsningar}
\label{sec:delimitations}

This is where the main delimitations are described. For
example, this could be that one has focused the study on a
specific application domain or target user group. In the
normal case, the delimitations need not be justified.

%\nocite{scigen}
%We have included Paper \ref{art:scigen}

%%%%%%%%%%%%%%%%%%%%%%%%%%%%%%%%%%%%%%%%%%%%%%%%%%%%%%%%%%%%%%%%%%%%%%
%%% Intro.tex ends here
