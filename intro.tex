\chapter{Inledning}
\label{cha:introduction}

Denna rapport redogör för hur ett system för 3D-kopiering har konstruerats. 
%% Vad är det för spännande projekt ni gjort?
%% Varför skall jag forsätta läsa rapporten?
\section{Motivering}
\label{sec:motivation}



Tekniken för att kunna skriva ut 3D-objekt har funnits i många år men först på 2010-talet har 3D-skrivare blivit tillgängliga även för vanliga konsumenter. Även i industrin har det blivit allt vanligare att använda 3D-skrivare för att skapa prototyper. Det finns dock ett problem: för att kunna använda 3D-skrivaren måste man antingen kunna CAD-mjukvara eller förlita sig på andra människors 3D-modeller. Genom att använda ett system för 3D-kopiering kan ett verkligt objekt istället kopieras.


%% Vad skall rapporten leda till? (Beskriva projekt, samla 
%% erfarenheter, speciella djupdykningar)
\section{Syfte}
\label{sec:aim}

Nedan beskrivs två syften med detta projekt. Det första syftet som beskrivs, vidareutveckling av TreeD, var syftet med projektet innan projektet reviderades medan det andra syftet, 3DCopy, var syftet med projektet efter revidering.

\subsection{Vidareutveckling av TreeD}
Syftet med detta projekt var till en början att konstruera en mjukvara för att styra ett system som kopierar tredimensionella objekt. Projektet skulle bygga vidare på ett tidigare kandidatarbete som utfördes våren 2016. Det kandidatarbete som genomfördes våren 2016 resulterade i en mjukvara som styr ett rotationsbord och en linjärenhet. På denna linjärenhet finns det en avståndskamera som används för att generera punktmoln. 

Syftet med vårt projekt var alltså att vidareutveckla resultatet från det tidigare kandidatprojektet till ett system som kan kopiera ett tredimensionellt objekt. Tanken var att detta skulle göras genom att ett flertal skanningar tas med olika rotationer och lutningar. De genererade punktmolnen ska sedan registreras till ett komplett punktmoln som sedan omvandlas till en 3D-mesh. Denna 3D-mesh kan sedan skrivas ut med hjälp av en 3D-skrivare.

\subsection{3DCopy}
Med revideringen av projektet reviderades även syftet. Det nya syftet är likt det gamla men försummar det tidigare projektets mjukvara helt. Detta betyder att systemet tar in ett antal punktmoln som registreras till ett komplett punktmoln som sedan omvandlas till en 3D-mesh. Denna 3D-mesh kan sedan skrivas ut med hjälp av en 3D-skrivare.

\subsection{GAMMALT}
Syftet med detta projekt var till en början att konstruera en mjukvara för att styra ett system som kopierar tredimensionella objekt. Projektet skulle bygga vidare på ett tidigare kandidatarbete som utfördes våren 2016. Resultatet av det tidigare projektet var en mjukvara som styr ett rotationsbord och en linjärenhet. Denna linjärenhet innehåller en avståndskamera som genererar ett punktmoln, se kapitel \ref{cha:theory}. Syftet var alltså att vidareutveckla resultatet från det tidigare projektet till ett system som kan kopiera ett tredimensionellt objekt. Tanken var att detta skulle göras genom att ett flertal skanningar tas med olika rotationer och lutningar, för att sedan registrera de resulterande punktmolnen till ett komplett punktmoln för objektet. Detta kompletta punktmoln skulle sedan omvandlas till en 3D-mesh, se avsnitt \ref{sec:definitions}, och sparas i ett format som kan användas för att skriva ut objektet på en 3D-skrivare.  Under projektets gång upptäcktes ett flertal problem med det tidigare projektets mjukvara som gjorde att målet med projektet reviderades. Det nya målet är likt det gamla men försummar det tidigare projektets mjukvara helt. Detta betyder att systemet tar in ett antal punktmoln som registreras till ett komplett punktmoln som sedan omvandlas till en 3D-mesh. Utifrån 3D-meshen genereras sedan G-code, kod för en 3D-skrivare, som kan föras över till en 3D-skrivare och skriva ut objektet.


\section{Frågeställning}
\label{sec:research-questions}
De generella samt specifika frågeställningar som denna rapport kommer behandla listas nedan.

\subsection{Generella frågeställningar}

\begin{enumerate}
	\item Hur kan ett system för 3D-kopiering implementeras så att man skapar värde för kunden?
	\item Vilka erfarenheter kan dokumenteras från ett 3D-kopieringsprojekt som kan vara intressanta för framtida projekt?
	\item Vilket stöd kan man få genom att skapa och följa upp en systemanatomi?
\end{enumerate}
	
\subsection{Specifika frågeställningar}

\begin{enumerate}
	\item [4.] Hur påverkas projektet av en helt reviderad kravspecifikation efter hälften av projektets gång?
	\item [5.] Hur har gruppens tillvägagångssätt ändrats på grund av faktumet att ett system skulle vidareutvecklas? 
	
\end{enumerate}

\section{Avgränsningar}
\label{sec:delimitations}
Denna rapport hade innan projektets revidering avgränsningar att systemet inte skulle utveckla det tidigare systemet TreeD även om behov skulle finnas. Det visade sig att behov fanns och därför reviderades projektet istället.

De nya avgränsningarna till projektet var att systemet enbart kommer behandla punktmoln tagna med den tillgängliga hårdvaran. Systemet ska enbart klara av att registrera dessa punktmoln och sedan generera en 3D-mesh för utskrift. Systemet ska inte behandla punktmolnen på något annat vis.

\section{Definitioner}
\label{sec:definitions}
Här definieras vissa begrepp och förkortningar som används senare i rapporten.

\begin{itemize}
	\item ICP - \textit{Iterative closest point}, en optimeringsalgoritm för att minimera avståndet mellan punkterna i två mängder av punkter. Vanligt använd för att registrera punktmoln.
	\item Point Cloud Library (PCL) - Ett C++-bibliotek för hantering av punktmoln.
	\item Robot Operating System (ROS) - En samling av ramverk för att utveckla mjukvara för robotar.
	\item CLI - Förkortning för det engelska uttrycket \textit{command line interface}, kommandoradsgränssnitt på svenska.
	\item GUI - Förkortning för det engelska uttrycket \textit{graphical user interface}, grafiskt användargränssnitt på svenska.
	\item 3DCopy - Programmet som utvec\part{title}klades i detta projekt.
	\item TreeD - Det befintliga systemet som gruppen skulle vidareutveckla innan omförhandling av projektet.
	\item Slack - Det kommunikationsverktyg som gruppen har använt under projektets gång.
	\item Trello - Verktyg som gruppen använt för att underlätta hantering av aktiviteter.
	\item Git - Versionshanteringssystem som gruppen har använt under projektets gång.
	\item 3D-mesh - En vattentät modell av ett objekt uppbyggt av polygoner.
	\item Voxel - Ett punktmoln kan delas in i så kallade Voxels vilket kan ses som kuber.
	\item Rotationsbord - En hårdvaruenhet som används för att rotera det objekt som ska skannas.
	\item Linjärenhet - En hårdvaruenhet som används för att förflytta avståndskameran i horisontalt plan.
	\item ROS-nod - En enskild process som kommunicerar med andra noder inom ramverket ROS.
	\item Nedsampling - Förkortning för det engelska uttrycket \textit{down sampling}. Uttrycket innebär att punkter kan tas bort ifrån ett punktmoln för att samtidigt uppskatta en ny yta som representerar den gamla så nära som möjligt, det är alltså en form av filtrering.
\end{itemize} 

%\nocite{scigen}
%We have included Paper \ref{art:scigen}

%%%%%%%%%%%%%%%%%%%%%%%%%%%%%%%%%%%%%%%%%%%%%%%%%%%%%%%%%%%%%%%%%%%%%%
%%% Intro.tex ends here
