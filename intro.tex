\chapter{Inledning}
\label{cha:introduction}

%% Vad är det för spännande projekt ni gjort?
%% Varför skall jag forsätta läsa rapporten?
\section{Motivering}
\label{sec:motivation}

Denna rapport redogör för hur ett system för 3D-kopiering har konstruerats. 

Tekniken för att kunna skriva ut 3D-objekt har funnits i många år men först på 2010-talet har 3D-skrivare kommit ner i pris så pass att vanliga konsumenter kan köpa en. Även i industrin har det blivit allt vanligare att använda 3D-skrivare för prototyper. Det finns dock ett problem: för att kunna använda 3D-skrivaren måste man antingen kunna CAD-mjukvara eller förlita sig på andra människors 3D-modeller. Genom att använda ett system för 3D-kopiering kan ett verkligt objekt istället kopieras.


%% Vad skall rapporten leda till? (Beskriva projekt, samla 
%% erfarenheter, speciella djupdykningar)
\section{Syfte}
\label{sec:aim}

Syftet med detta projekt var tidigt i projektet att konstruera en mjukvara för att styra ett system som kopierar tredimensionella objekt. Projektet skulle bygga vidare på ett tidigare kandidatarbete som utfördes våren 2016. Resultatet av det tidigare projektet var en mjukvara som styr ett rotationsbord och en linjärenhet. Denna linjärenhet styr i sig en avståndskamera som sedan genererar ett punktmoln (se kap. \ref{cha:theory}). Syftet var alltså att vidareutveckla resultatet från det tidigare projektet till ett system som kan kopiera ett tredimensionellt objekt. Tanken var att detta skulle göras genom att ett flertal skanningar tas med olika rotationer och lutningar för att sedan registrera de resulterande punktmolnen till ett komplett punktmoln för objektet. Detta kompletta punktmoln skulle sedan omvandlas till en 3D-mesh (se kap. \ref{sec:definitions}) och sparas i ett format som kan användas för att skriva ut objektet på en 3D-skrivare.  Under projektets gång upptäcktes ett flertal problem med det tidigare projektets mjukvara som gjorde att målet med projektet reviderades. Det nya målet är att likt det gamla men försummar det tidigare projektets mjukvara helt. Detta betyder att systemet tar in ett antal punktmoln som registreras till ett komplett punktmoln som sedan omvandlas till en 3D-mesh.


\section{Frågeställning}
\label{sec:research-questions}
De generella samt specifika frågeställningar som denna rapport kommer ta upp listas nedan.

\subsection{Generella frågeställningar}

\begin{enumerate}
	\item Hur kan ett system för 3D-kopiering implementeras så att man skapar värde för kunden?
	\item Vilka erfarenheter kan dokumenteras från ett 3D-kopieringsprojekt som kan vara intressanta för framtida projekt?
	\item Vilket stöd kan man få genom att skapa och följa upp en systemanatomi?
	
\subsection{Specifika frågeställningar}

	\item Hur påverkas projektet av en helt reviderad kravspecifikation efter hälften av projektets gång?
	\item Hur har gruppens tillvägagångssätt ändrats på grund av faktumet att vi vidareutvecklade ett system? 
	
\end{enumerate}

\section{Avgränsningar}
\label{sec:delimitations}
Denna rapport hade innan projektets revidering avgränsningar att systemet inte skulle utveckla det tidigare systemet TreeD även om behov skulle finnas. Det visade sig att behov fanns och därför revideras projektet istället.

De nya avgränsningarna till projektet är att systemet enbart kommer behandla punktmoln tagna med den tillgängliga hårdvaran och inga andra punktmoln. Systemet ska enbart klara av att registrera dessa punktmoln och sedan generera en 3D-mesh för utskrift. Systemet ska inte behandla punktmolnen på något annat vis.

Gruppen har 2800 timmar till sitt förfogande och inom dessa timmar ska systemet utvecklas och alla projektrelaterade dokument skrivas.

\section{Definitioner}
\label{sec:definitions}
Här definieras vissa begrepp och förkortningar som används senare i rapporten.

\begin{itemize}
	\item ICP - \textit{Iterative closest point}, en optimeringsalgoritm för att minimera avståndet mellan punkterna i två mängder av punkter. Vanligt använd för att registrera punktmoln.
	\item CLI - Förkortning för det engelska uttrycket \textit{command line interface}, kommandoradsgränssnitt på svenska.
	\item 3DCopy - Programmet som utvecklades i detta projekt
	\item TreeD - Det befintliga programmet som gruppen skulle vidareutveckla innan omförhandling av projektet.
\end{itemize} 

%\nocite{scigen}
%We have included Paper \ref{art:scigen}

%%%%%%%%%%%%%%%%%%%%%%%%%%%%%%%%%%%%%%%%%%%%%%%%%%%%%%%%%%%%%%%%%%%%%%
%%% Intro.tex ends here
