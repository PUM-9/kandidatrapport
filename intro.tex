\chapter{Inledning}
\label{cha:introduction}

%% Vad är det för spännande projekt ni gjort?
%% Varför skall jag forsätta läsa rapporten?
\section{Motivering}
\label{sec:motivation}

Denna rapport redogör för hur ett system för 3D-kopiering har konstruerats. 

Tekniken för att kunna skriva ut 3D-objekt har funnits i många år men först på 2010-talet har 3D-skrivare kommit ner i pris så pass att vanliga konsumenter kan köpa en. Även i industrin har det blivit allt vanligare att använda 3D-skrivare för prototyper. Det finns dock ett problem: för att kunna använda 3D-skrivaren måste man antingen kunna CAD-mjukvara eller förlita sig på andra människors 3D-modeller. Genom att använda ett system för 3D-kopiering kan ett verkligt objekt istället kopieras.


%% Vad skall rapporten leda till? (Beskriva projekt, samla 
%% erfarenheter, speciella djupdykningar)
\section{Syfte}
\label{sec:aim}

Syftet med detta projekt var att konstruera en mjukvara för att styra ett system som kopierar tredimensionella objekt. Projektet bygger vidare på ett tidigare kandidatarbete som utfördes våren 2016. Resultatet av det tidigare projektet var en mjukvara som styr ett rotationsbord och en linjärenhet med avståndskamera och genererar ett punktmoln (se kap. \ref{cha:theory}). Syftet var alltså att vidareutveckla resultatet från det tidigare projektet till ett system som kan kopiera ett tredimensionellt objekt. Detta görs genom att ett flertal skanningar tas med olika rotationer och lutningar för att sedan registrera de resulterande punktmolnen till ett komplett punktmoln för objektet. Detta kompletta punktmoln omvandlas sedan till en 3D-mesh (se definitioner) och sparas i ett format som kan användas för att skriva ut objektet på en 3D-skrivare. Systemet är modulärt, välbyggt och enkelt att förstå för att göra det möjligt för vidareutveckling, forskning samt för eventuell användning till laborationer.   


\section{Frågeställning}
De generella samt specifika frågeställningar som denna rapport kommer ta upp listas nedan.

\subsection{Generella frågeställningar}
\label{sec:research-questions-general}

\begin{enumerate}
	\item Hur kan ett system för 3D-kopiering implementeras så att man skapar värde för kunden?
	\item Vilka erfarenheter kan dokumenteras från ett 3D-kopieringsprojekt som kan vara intressanta för framtida projekt?
	\item Vilket stöd kan man få genom att skapa och följa upp en systemanatomi?
	
\subsection{Specifika frågeställningar}
\label{sec:research-questions-specific}
	
	\item Hur påverkas projektet av en helt reviderad kravspecifikation efter hälften av projektets gång?
	\item Hur har gruppens tillvägagångssätt ändrats på grund av faktumet att vi vidareutvecklade ett system? 
	
\end{enumerate}

\section{Definitioner}
\label{sec:definitions}
Här definieras vissa begrepp och förkortningar som används senare i rapporten.

\begin{itemize}
\item ICP - \textit{Iterative closest point}, en optimeringsalgoritm för att minimera avståndet mellan en mängd punkter. Vanligt använd för att registrera punktmoln. 
\end{itemize} 

\section{Avgränsningar}
\label{sec:delimitations}

Denna rapport behandlar utveckling av ett system för registrering av punktmoln, skapande av mesh samt skapande av kod för 3D-skrivare. Utvecklingen sker med förutsättningen att man har tillgång till TreeD, ett system utvecklat av en tidigare ett tidigare kandidatprojekt, samt den hårdvara som specificeras i rapporten. Tanken är att systemet ska fungera med vilket annat system som helst istället för TreeD som kan tillhandahålla punktmoln men det finns inga garantier för att fallet skulle vara så. Gruppen har 2800 timmar till sitt förfogande och inom dessa timmar ska systemet utvecklas och alla projektrelaterade dokument skrivas. 


%\nocite{scigen}
%We have included Paper \ref{art:scigen}

%%%%%%%%%%%%%%%%%%%%%%%%%%%%%%%%%%%%%%%%%%%%%%%%%%%%%%%%%%%%%%%%%%%%%%
%%% Intro.tex ends here
