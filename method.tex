%% 
\chapter{Metod}
\label{cha:method}

\section{Utvecklingsmetod}
I projektet har man försökt använda en iterativ process de vissa inslag av vattenfallsmodellen pågrund av den korta tidsperioden under vilken projektet ska utföras. Utvecklingsledaren har varit ansvarig för att ta fram vad som ska göras i varje iteration och planera iterationerna i samrådan med övriga gruppmedlemmar. Till exempel har teamleader varit ansvarig för att planera de administrativa delarna. För att planera en iteration så utgick man från kravspecifikationen och bröt ner den till aktiviteter. Där varje aktivitet ska vara genomförbar på mindre än en vecka. Det var iallafall målet men sen blev det inte alltid som man planerat. När aktiviteterna var skapade så gjorde utvecklingsledaren med inrådan från bland annat arkitekten en tidsuppskattning på varje aktivitet och hur många som borde jobba med aktiviteten. Efter planneringen gick man igenom aktiviteterna med gruppen. När alla vet vad som ska göras under iterationen väljer gruppmedlemmarna vilka aktiviteter de vill detla i. Vidare under iterationen planeras vad som ska göras i varje aktivitet av de som är involverade i den aktiviteten.

\subsection{Förstudie}
Projektet startades med en förstudie där gruppen studerade de områden, verktyg och ramverk som man visste skulle behövas för projektet. Informationen man hade från början var den kunden tillhandahållit därifrån upptäcktes en hel del nya områden som behövde undersökas. Under förstudien delade utvecklingsledaren ut olika områden att undersöka. Efter något undersökts skrevs ett kort dokument som beskrev slutsatserna och relevanta länkar från undersökningen. De största områdena som undersöktes var ROS och PCL. Framförallt ROS som ingen i gruppen hade arbetat med tidigare och som har en stor påverkan på hur arkitekturen och implementation skulle genomföras. Eftersom det hade så stor påverkan så avslutade den undersökningen med en lista av guider alla gruppmedlemmar genomförde. När alla genomfört de guiderna gjordes en kodutmaning där alla gruppmedlemmar skrev en chattklient med hjälp av ROS. Detta avslutade förstudie fasen med att alla fick skriva lite kod vilket var uppskattat. 

\subsection{Iteration 1}
En två veckor lång iteration med drömläget att få klart basfunktionaliteten för produkten. Första veckan hade fokus på att skriva kod, skapa ett interface till den existerande mjukvaran, registrera och mesha punktmoln. Andra veckan var det mer fokus på dokumentation, samt att färdigställa så mycket som möjligt av det som påbörjats första veckan. Målet var nog lite högt satt men resultatet blev tillfredställande.

\section{Metod för att fånga erfarenheter}

In this chapter, the method is described in a way which shows how the
work was actually carried out. The description must be precise and
well thought through. Consider the scientific term
replicability. Replicability means that someone reading a scientific
report should be able to follow the method description and then carry
out the same study and check whether the results obtained are
similar. Achieving replicability is not always relevant, but precision
and clarity is.

Sometimes the work is separated into different parts, e.g.  pre-study,
implementation and evaluation. In such cases it is recommended that
the method chapter is structured accordingly with suitable named
sub-headings.

%%%%%%%%%%%%%%%%%%%%%%%%%%%%%%%%%%%%%%%%%%%%%%%%%%%%%%%%%%%%%%%%%%%%%%
%%% method.tex ends here
