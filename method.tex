%% 
\chapter{Metod}
\label{cha:method}

\section{Utvecklingsmetod}

I projektet har man försökt använda en iterativ process med vissa inslag av vattenfallsmodellen på grund av den korta tidsperioden under vilken projektet skulle utföras. Utvecklingsledaren har varit ansvarig för att ta fram vad som ska göras i varje iteration och planera iterationerna i samråd med övriga gruppmedlemmar. Till exempel har teamleader varit ansvarig för att planera de administrativa delarna. För att planera en iteration så utgick man från kravspecifikationen och bröt ner den till aktiviteter. Där varje aktivitet ska vara genomförbar på mindre än en vecka. Detta var tanken i teorin men i praktiken höll inte alltid planeringen. När aktiviteterna var skapade så gjorde utvecklingsledaren med inrådan från bland annat arkitekten en tidsuppskattning på varje aktivitet och hur många som borde jobba med aktiviteten. Efter planeringen gick man igenom aktiviteterna med gruppen. När alla vet vad som ska göras under iterationen väljer gruppmedlemmarna vilka aktiviteter de vill delta i. Vidare under iterationen planeras vad som ska göras i varje aktivitet av de som är involverade i den aktiviteten.

\subsection{Förstudie}

Projektet startades med en förstudie där gruppen studerade de områden, verktyg och ramverk som man visste skulle behövas för projektet. Informationen man hade från början var den kunden tillhandahållit därifrån upptäcktes en del nya områden som behövde undersökas. Under förstudien delade utvecklingsledaren ut olika områden att undersöka. Efter något undersökts skrevs ett kort dokument som beskrev slutsatserna och relevanta länkar från undersökningen. De största områdena som undersöktes var ROS och PCL. Framförallt ROS som ingen i gruppen hade arbetat med tidigare och som har en stor påverkan på hur arkitekturen och implementation skulle genomföras. Eftersom det hade så stor påverkan så avslutade den undersökningen med en lista av guider alla gruppmedlemmar genomförde. När alla genomfört de guiderna gjordes en kodutmaning där alla gruppmedlemmar skrev en chattklient med hjälp av ROS. Detta avslutade förstudie fasen med att alla fick skriva lite kod. 

\subsection{Iteration 1}

En två veckor lång iteration med drömläget att få klart basfunktionaliteten för produkten. Första veckan hade fokus på att skriva kod, skapa ett interface till den existerande mjukvaran, registrera och mesha punktmoln. Andra veckan var det mer fokus på dokumentation, samt att färdigställa så mycket som möjligt av det som påbörjats första veckan. Målet var nog lite högt satt men resultatet blev tillfredsställande.

\section{Metod för att fånga erfarenheter}

Erfarenheter fångades upp under projektets gång med hjälp av utvärderingar och undersökningar efter varje iteration. I dessa så fick alla gruppmedlemmar svara på en del frågor kring hur arbetet gått, vad som varit svårt och vad som gått bra. De två fasta mötena varje vecka har också varit ett forum att dela med sig och fånga upp erfarenheter. Den här rapporten har också används för att samla ihop och skriva ner erfarenheter från projektet. 


%%%%%%%%%%%%%%%%%%%%%%%%%%%%%%%%%%%%%%%%%%%%%%%%%%%%%%%%%%%%%%%%%%%%%%
%%% method.tex ends here
