%% 
\chapter{Metod}
\label{cha:method}

\section{Utvecklingsmetod}
I projektet har vi försökt använda oss av en iterativ process de vissa inslag av vattenfallsmodellen pågrund av den korta tidsperioden under vilken projektet ska utföras. Utvecklingsledaren har varit ansvarig för att ta fram vad som ska göras i varje iteration 

\section{Metod för att fånga erfarenheter}

In this chapter, the method is described in a way which shows how the
work was actually carried out. The description must be precise and
well thought through. Consider the scientific term
replicability. Replicability means that someone reading a scientific
report should be able to follow the method description and then carry
out the same study and check whether the results obtained are
similar. Achieving replicability is not always relevant, but precision
and clarity is.

Sometimes the work is separated into different parts, e.g.  pre-study,
implementation and evaluation. In such cases it is recommended that
the method chapter is structured accordingly with suitable named
sub-headings.

%%%%%%%%%%%%%%%%%%%%%%%%%%%%%%%%%%%%%%%%%%%%%%%%%%%%%%%%%%%%%%%%%%%%%%
%%% method.tex ends here