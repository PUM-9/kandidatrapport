%% 
\chapter{Metod}
\label{cha:method}

\section{Utvecklingsmetodik}

I projektet har man försökt använda en iterativ process med vissa inslag av vattenfallsmodellen på grund av den korta tidsperioden som projektet skulle genomföras under. Utvecklingsledaren har varit ansvarig för att ta fram vad som ska göras i varje iteration och planera iterationerna i samråd med övriga gruppmedlemmar. Teamledaren har varit ansvarig för att planera de administrativa delarna av projektet. För att planera en iteration så utgick man från kravspecifikationen och bröt ner den till aktiviteter. Målet med varje skapad aktivitet var att den ska vara genomförbar på mindre än en vecka. Detta var tanken i teorin men i praktiken höll inte alltid planeringen. När aktiviteterna var skapade så gjorde utvecklingsledaren med inrådan från bland annat arkitekten en tidsuppskattning på varje aktivitet och hur många som borde jobba med aktiviteten. Efter planeringen gick man igenom aktiviteterna med gruppen. När alla vet vad som ska göras under iterationen väljer gruppmedlemmarna vilka aktiviteter de vill delta i. Vidare under iterationen planeras vad som ska göras i varje aktivitet av de som är involverade i den aktiviteten.

\subsection{Förstudie}

Projektet startades med en förstudie där gruppen sammanställde diverse dokument. Syftet med dessa dokument är att både gruppmedlemmarna själva och externa parter ska få en bättre insikt i vad projektet går ut på och vad dess mål är. Dokumenten utformades och togs fram av den eller de personer i gruppen som var bäst lämpad för ett visst dokument, med det menas den person som passade bäst för ett visst dokument med avseende på dennes roll. Detta ledde även till att den personen fick mer erfarenheter och kunskaper om vad dennes roll innebar.

Gruppen började sedan att studera de områden, verktyg och ramverk som man visste skulle behövas för projektet. Den information som gruppen hade från början var den som kunden tillhandahållit och utifrån denna information upptäcktes en del nya områden som behövde undersökas. Under förstudien delade utvecklingsledaren ut olika områden att undersöka. Efter ett område undersökts skrevs ett kort dokument som beskrev slutsatserna och relevanta länkar från undersökningen. De största områdena som undersöktes var ROS och PCL. Framförallt undersöktes ROS eftersom det hade en stor påverkan på hur arkitekturen och implementation skulle genomföras. Att undersöka ROS noggrant var dessutom välbehövligt eftersom ingen i gruppen hade arbetat med det tidigare. Eftersom det hade så stor påverkan så avslutades den undersökningen med en lista av guider som alla gruppmedlemmar genomförde. När alla genomfört de guiderna genomfördes en kodutmaning där alla gruppmedlemmar skrev en chattklient med hjälp av ROS. Detta avslutade förstudiefasen med att alla fick skriva lite kod.

\subsection{Iteration 1}

En två veckor lång iteration där målet är att få klart basfunktionaliteten för produkten. Första veckan hade fokus på att skriva kod, skapa ett gränssnitt till den existerande mjukvaran, registrera och mesha punktmoln. Andra veckan var det mer fokus på dokumentation, samt att färdigställa så mycket som möjligt av det som påbörjades under första veckan.

\subsection{Iteration 2}

Iterationen inleddes med fokus på registrering då detta var det stora frågetecknet i projektet. Efter att först ha undersökt olika algoritmer beslutades det att vi skulle fortsätta med ICP då den hade bäst stöd i PCL-biblioteket. Efter det så delades gruppen i två för att arbeta med två olika algoritmer för att registrera punktmolnen. Den första algoritmen var att försöka vrida och placera punktmolnen rätt från början och på så sätt få det väldigt enkelt att sedan registrera punktmolnen med ICP. Problemet med att få algoritmen att fungera var att den behövde vara väldigt generell vilket gjorde den svår att ta fram och fullända. Speciellt för de då ansedda enkla objekten som var väldigt släta. Den andra algoritmen var mer anpassad för de då ansedda enkla objekten. Den gick ut på att i första hand endast lösa problemet med att registrera rätblocksliknande föremål. Detta gjorde den genom att hitta en så platt yta som möjligt, anta att det var en sida och sedan ta fyra skanningar med 90 graders rotation mellan varje. Dessa punktmoln registrerades sedan på ett genererat rätblock med ICP.

Den andra algoritmen var den som utforskades mest tills den skulle börja testas ihop med hårdvaran. Då upptäcktes att det fanns mängder med fel på det underliggande systemet som tog fram punktmolnen. Det gick inte att genomföra tillräckligt många skanningar i rad utan att det underliggande systemet kraschade. 

Detta gjorde att vi tillsammans med kunden fick göra om en stor del av projektet där vi bortsåg från det tidigare systemet helt och hållet. Vi beslutade också tillsammans med kunden att det borde vara lättare att registrera mer detaljerade objekt. Vi gick då tillbaka till algoritmen för att placera rätt och sedan registrera. Denna nya registrering placerades i ett helt nytt program så vi började även bygga ett CLI och integrera meshning och registrering i ett enda program. Istället för att som tidigare bygga allt som enskilda ROS-noder.

\subsection{Iteration 3}

Den första veckan låg fokus på de individuella delarna av kandidatrapporten men det arbetades även en del med den gemensamma rapporten. Rapporten lämnades sedan in i slutet av första veckan. Andra veckan låg fokus på den gemensamma delen av rapporten. En del komplettering av de individuella delarna gjordes också då vi fick feedback på första veckans inlämning redan i början av andra veckan.


\section{Metod för att fånga erfarenheter}

Erfarenheter fångades upp under projektets gång med hjälp av kontinuerlig utvärdering och diskussioner under gruppens gemensamma möten. Dessa möten skedde två gånger i veckan, varje måndag och torsdag. På måndagar deltog även handledaren och då låg fokus på en statusuppdatering och vad som skulle göras i veckan. Under statusuppdateringen diskuterades vad som gått bra, vad som gått dåligt och hur vi skulle kunna förbättra saker och ting. Under torsdagsmöten las fokus på diskussion kring viktiga frågor som hela gruppen behövde vara del av. Erfarenheterna som kommit upp på dessa möten, framförallt måndagsmötena sammanfattades av teamledaren i varje veckas statusrapport.

Erfarenheter har också delats av gruppen under arbetets gång. Då vi arbetat större delen av tiden i samma rum har det skett mycket diskussion i gruppen kring olika problem öppet i rummet. Slutligen har vi samlat våra erfarenheter i den här rapporten som skrivits under projektets gång och därav fångat upp erfarenheter allt eftersom de dykt upp.


%%%%%%%%%%%%%%%%%%%%%%%%%%%%%%%%%%%%%%%%%%%%%%%%%%%%%%%%%%%%%%%%%%%%%%
%%% method.tex ends here
