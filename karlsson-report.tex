\chapter{Analys av punktmolnsregistrering}
\label{cha:indiv-report-karlsson}
\chapterprecis{\LARGE{---- Michael Karlsson ----}}


\section{Inledning}
\label{sec:introduction-karlsson}

%% Skriv här
I denna del behandlas Michael Karlssons undersökning av olika registreringsalgoritmer och de problem gruppen stött på i samband med registrering av punktmoln.

\subsection{Syfte}
\label{sec:purpose-karlsson}

%% Skriv här
Syftet med denna delen är att väga olika algoritmer mot varandra och hur väl de fungerade för de projekt som beskrivs i rapporten samt vilka problem gruppen hade med de olika algoritmer som testades. Gruppens tillvägagångssätt för val av algoritm undersöks också.


\subsection{Frågeställning}
\label{sec:issue-karlsson}

\begin{itemize}
	\item Hur skapar man ett enhetligt punktmoln från enstaka bilder från en fast \newline avståndskamera?
\end{itemize}
\subsection{Avgränsningar}

Denna rapport kommer begränsas till hur processen fungerat med den hårdvara som vi blivit tilldelade.


\section{Bakgrund}
\label{sec:background-karlsson}

\subsection{Registrering}

\subsection{Kinect Fusion}

Kinect fusion använder sig av Microsofts egna avstånds- och RGB-kamera för att mappa upp ett 3d objekt i den verkliga världen. 

\section{Teori}
\label{sec:theory-karlsson}

När man vill återskapa fysiska objekt digitalt fastnar man oftast i att det inte finns några bra verktyg för att läsa in 3-dimensionella objekt. Man behöver då ta flera bilder av objektet från olika vinklar för att kunna sammanfoga dessa till det fulla objektet. Det finns många metoder för detta, exempelvis Kinect Fusion. Den metod vi använt oss av har dock bestått av en avståndskamera på linjärenhet med objektet monterat på ett rotationsbord med 2 rotationsaxlar. Med detta rotationsbord kan men förhållandevis enkelt se hela objektet från en fast punkt.

\section{Metod}
\label{sec:method-karlsson}


Metoden för att samla in data kommer i första hand ske genom den undersökning som gruppen gör på ämnet. Vid behov kommer en intervjustudie i ämnet göras internt i gruppen. 


\section{Resultat}
\label{sec:results-karlsson}

%% Skriv här

\section{Diskussion}
\label{sec:discussion-karlsson}

%% Skriv här

\section{Slutsatser}
\label{sec:conclusions-karlsson}

%% Skriv här

%%%%%%%%%%%%%%%%%%%%%%%%%%%%%%%%%%%%%%%%%%%%%%%%%%%%%%%%%%%%%%%%%%%%%%
%%% karlsson-report.tex ends here
