\chapter{Testning av ett system byggt på ROS av Olof Holmberg}
\label{cha:indiv-report-person}

\section{Inledning}
\label{sec:introduction-person}

Testning av mjukvara är viktigt för att hitta eventuella buggar samt säkerställa kvalité och kravuppfyllnad hos mjukvaran. Men komplexiteten av testningen ökar när antalet enheter i systemet ökar. I den här rapporten kommer testning av ett system som bygger på ROS att undersökas. Vilka tester bör man använda? Hur testar man ett system med flera noder som kommunicerar?

\subsection{Syfte}
\label{sec:purpose-person}

Syftet med denna rapport är att utforska de tekniker som man kan använda sig av när man testar ett system som bygger på ROS. Den undersöker också vilka testtekniker som rekommenderas och tar fram riktlinjer för hur man bör gå tillväga när man testar ett system byggt på ROS.

\subsection{Frågeställning}
\label{sec:issue-person}

\begin{itemize}
	\item Hur testar man ett system som bygger på ROS?
	\item Hur påverkas utvecklingen av systemet utifrån testningen av systemet?
\end{itemize}

\subsection{Eventuella tänkta referenser}

Rapporten kommer att samla information utifrån artiklar som är publicerade på konferenser. De artiklar som är av intresse är de som tar upp testning av ett system byggt på ROS samt de som tar upp testning av system med flera separata komponenter. Nedan listas de artiklar och källor som eventuellt kommer att användas till rapporten.

\begin{itemize}
	\item [1] \textquotedblright Documentation - ROS Wiki," in ROS.org, 2017. [Online]. Available: \url{http://wiki.ros.org}. Accessed: Mar. 06, 2017.
	\item [2] J. Ernits, E. Halling, G. Kanter, J. Vain, \textquotedblright Model-based integration testing of ROS packages:
	a mobile robot case study", 2015. [IEEE]. Available: \url{http://ieeexplore.ieee.org/stamp/stamp.jsp?arnumber=7324210}.	
\end{itemize}

\section{Bakgrund}
\label{sec:background-person}

%% Skriv här

\section{Teori}
\label{sec:theory-person}

%% Skriv här

\section{Metod}
\label{sec:method-person}

%% Skriv här

\section{Resultat}
\label{sec:results-person}

%% Skriv här

\section{Diskussion}
\label{sec:discussion-person}

%% Skriv här

\section{Slutsatser}
\label{sec:conclusions-person}

%% Skriv här

%%%%%%%%%%%%%%%%%%%%%%%%%%%%%%%%%%%%%%%%%%%%%%%%%%%%%%%%%%%%%%%%%%%%%%
%%% person-report.tex ends here
