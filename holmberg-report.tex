\chapter{Testning av ett system byggt på ROS}
\label{cha:indiv-report-person}

\section{Inledning}
\label{sec:introduction-person}

Testning av mjukvara är viktigt för att hitta eventuella buggar samt säkerställa kvalité och kravuppfyllnad hos mjukvaran. Men komplexiteten av testningen ökar när antalet enheter i systemet ökar. I den här rapporten kommer testning av ett system som bygger på ROS att undersökas. Vilka tester bör man använda? Hur testar man ett system med flera noder som kommunicerar?

\subsection{Syfte}
\label{sec:purpose-person}

Syftet med denna rapport är att utforska de tekniker som man kan använda sig av när man testar ett system som bygger på ROS. Den undersöker också vilka testtekniker som rekommenderas och tar fram riktlinjer för hur man bör gå tillväga när man testar ett system byggt på ROS.

\subsection{Frågeställning}
\label{sec:issue-person}

\begin{itemize}
	\item Hur testar man ett system som bygger på ROS?
	\item Hur påverkas utvecklingen av systemet utifrån testningen av systemet?
\end{itemize}

\section{Bakgrund}
\label{sec:background-person}

%% Skriv här

\section{Teori}
\label{sec:theory-person}

%% Skriv här

\section{Metod}
\label{sec:method-person}

%% Skriv här

\section{Resultat}
\label{sec:results-person}

%% Skriv här

\section{Diskussion}
\label{sec:discussion-person}

%% Skriv här

\section{Slutsatser}
\label{sec:conclusions-person}

%% Skriv här

%%%%%%%%%%%%%%%%%%%%%%%%%%%%%%%%%%%%%%%%%%%%%%%%%%%%%%%%%%%%%%%%%%%%%%
%%% person-report.tex ends here
