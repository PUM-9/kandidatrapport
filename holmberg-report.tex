\chapter{}
\chapterprecis{\LARGE{---- Olof Holmberg ----}}
\label{cha:indiv-report-holmberg}

\section{Inledning}
\label{sec:introduction-holmberg}

När man interagerar med mjukvara så gör man det oftast genom ett grafiskt användargränssnitt. När man interagerar så uppstår också en sekvens av interaktioner som bara blir längre och längre under tiden som man interagerar med mjukvaran. Då varje interaktion påverkar mjukvaran olika kan det uppstå fel som beror på sekvensen av interaktioner. Dessa fel behöver nödvändigtvis inte uppstå förrän sekvensen består av flera hundra olika interaktioner.

Detta gör att det är väldigt svårt att testa ett GUI. Det är nästintill omöjligt att testa alla olika kombinationer av interaktioner som användarna kommer att utsätta mjukvaran för. Därför är det viktigt att vid test av GUI:t ta fram testfall som testar så många olika sekvenser som möjligt utan att det tar för lång tid. Den här utredningen undersöker hur man kan ta fram de sekvenser som är mest kritiska att testa samt hur det tillämpades på projektet.

\subsection{Syfte}
\label{sec:purpose-holmberg}

Syftet med den här utredningen är att undersöka hur man tar fram de sekvenser som är viktigast att testa för att maximera testningen. Utredningen kommer också att undersöka för vilka GUI:n som den här formen av testning lämpar sig för samt hur gruppen använde sig av den under testning av projektets GUI.

\subsection{Frågeställning}
\label{sec:issue-holmberg}

\begin{itemize}
	\item [1] Hur tar man fram sekvenser av interaktioner som maximerar testningen av GUI:t?
	\item [2] Fungerar denna metod för alla typer av GUI:n? 
	\item [3] Vilka testfall/sekvenser användes under de test som utfördes på projektets GUI?
\end{itemize}

\subsection{Definitioner, akronym och förkortningar}
Följande definitioner och förkortningar används på flera ställen i denna del av rapporten:
\begin{itemize}
	\item GUI (Graphical user interface) - Det grafiska användargränssnittet som används för att interagera med mjukvaran.
\end{itemize}

\subsection{Eventuella tänkta referenser}

Rapporten kommer att samla information utifrån artiklar som är publicerade på konferenser. De artiklar som är av intresse är de som tar upp testning relaterat till användargränssnitt samt användarvänlighet hos grafiska användargränssnitt. Något som också är av intresse är hur vi i gruppen testar vårt gränssnitt och hur vi ser till att det är användarvänligt.

\begin{itemize}
	\item [1] X. Yuan, M.B. Cohen, A.M. Memon \textquotedblright GUI Interation Testing: Incorporating Event Context", 2011. [IEEE]. Available: \url{http://ieeexplore.ieee.org/abstract/document/5444885}.	
\end{itemize}


\section{Bakgrund}
\label{sec:background-holmberg}

%% Skriv här

\section{Teori}
\label{sec:theory-holmberg}

%% Skriv här

\section{Metod}
\label{sec:method-holmberg}

%% Skriv här

\section{Resultat}
\label{sec:results-holmberg}

%% Skriv här

\section{Diskussion}
\label{sec:discussion-holmberg}

%% Skriv här

\section{Slutsatser}
\label{sec:conclusions-holmberg}

%% Skriv här

%%%%%%%%%%%%%%%%%%%%%%%%%%%%%%%%%%%%%%%%%%%%%%%%%%%%%%%%%%%%%%%%%%%%%%
%%% person-report.tex ends here
