\chapter{Diskussion}
\label{cha:discussion}

This chapter contains the following sub-headings.

\section{Results}
\label{sec:discussion-results}

Are there anything in the results that stand out and need be
analyzed and commented on? How do the results relate to the
material covered in the theory chapter? What does the theory
imply about the meaning of the results? For example, what
does it mean that a certain system got a certain numeric value
in a usability evaluation; how good or bad is it? Is there
something in the results that is unexpected based on the
literature review, or is everything as one would theoretically
expect?

\section{Method}
\label{sec:discussion-method}

This is where the applied method is discussed and criticized.
Taking a self-critical stance to the method used is an
important part of the scientific approach.

A study is rarely perfect. There are almost always things one
could have done differently if the study could be repeated or
with extra resources. Go through the most important
limitations with your method and discuss potential
consequences for the results. Connect back to the method
theory presented in the theory chapter. Refer explicitly to
relevant sources.

The discussion shall also demonstrate an awareness of methodological
concepts such as replicability, reliability, and validity. The concept
of replicability has already been discussed in the Method chapter
(\ref{cha:method}). Reliability is a term for whether one can expect
to get the same results if a study is repeated with the same method. A
study with a high degree of reliability has a large probability of
leading to similar results if repeated. The concept of validity is,
somewhat simplified, concerned with whether a performed measurement
actually measures what one thinks is being measured. A study with a
high degree of validity thus has a high level of credibility. A
discussion of these concepts must be transferred to the actual context
of the study.

The method discussion shall also contain a paragraph of
source criticism. This is where the authors’ point of view on
the use and selection of sources is described.

In certain contexts it may be the case that the most relevant
information for the study is not to be found in scientific
literature but rather with individual software developers and
open source projects. It must then be clearly stated that
efforts have been made to gain access to this information,
e.g. by direct communication with developers and/or through
discussion forums, etc. Efforts must also be made to indicate
the lack of relevant research literature. The precise manner
of such investigations must be clearly specified in a method
section. The paragraph on source criticism must critically
discuss these approaches.

Usually however, there are always relevant related research.
If not about the actual research questions, there is certainly
important information about the domain under study.

\section{Arbetet i vidare sammanhang}
\label{sec:work-wider-context}

% Diskutera etiska och samhälleliga aspekter relaterade till arbetet. 
% Om arbetet av någon anledning, helt saknar en koppling till etiska eller samhälleliga aspekter anges och motiveras det i avgränsningar i introduktionskapitlet.
% Hänvisa till källor relevanta för diskussionen.

\subsection{Hållbar utveckling}
\label{disc:hållbar_utveckling}
Ett system för 3D-kopiering har många positiva effekter på samhället. Systemet gynnar framförallt miljön eftersom att tillverka en önskad produkt istället för att köpa en fabrikstillverkad sparar avsevärt på naturens tillgångar. Dels kommer 3D-kopieringssystemet att använda mindre energi och dessutom kommer transporterna till och från affären att minska, vilket leder till mindre koldioxidutsläpp. Kreiger \cite{kreiger2013environmental} har gjort en studie där han skriver ut 3D-produkter i plast och jämför kostnaden för att tillverka produkter av plast i en fabrik. Med kostnaden menas den energi som går åt från råmaterial till färdig produkt samt kostnaden som går åt för transport. Det visar sig att tillverka en produkt i en 3D-skrivare kräver mellan 41 till 64 procent mindre energi än att fabrikstillverka produkten. Förklaringen till detta är att produkter som skrivs ut i en 3D-skrivare kan göras mer ihåliga och således kräver de också mindre material. 

Det finns dessutom relaterade studier som visar att det blir billigare att skriva ut en produkt i en 3D-skrivare istället för att köpa en fabrikstillverkad \cite{wittbrodt2013life}. Det här främjar samhället i positiv beaktning eftersom det helt enkelt blir billigare för konsumenter att införskaffa sig de produkter de önskar. 

\subsection{Specifika förbättringspunkter till vårat system}
Vårt system för 3D-kopiering är som tidigare nämnts ett generellt bra system för att främja hållbar utveckling. Det finns däremot en del aspekter som skulle kunna gjorts annorlunda vid systemets uppbyggnad för att ytterligare främja hållbar utveckling. För att utveckla detta har vi valt att titta på hur våra krav framställdes, närmare bestämt besvara dessa frågor:

\begin{itemize}
	\item Hur skulle vi kunna göra annorlunda i kravprocessen?
	\item Vad har vi tagit hänsyn till i kravprocessen?
	\item Hur kan vi bedöma de krav vi satt på systemet med hållbar utveckling i åtanke?
\end{itemize}
Vid framställning av kraven till systemet fanns det inga tankar på att ta fram krav som främjar hållbar utveckling, det på grund av att ingen i projektgruppen hade erfarenheter från hållbar utveckling tidigare och således inte någon tanke på det. Vid framtagning av krav till systemet har alltså ingen hänsyn tagits till att främja hållbar utveckling. Det som har tagits hänsyn till i kravprocessen är endast funktionaliteten av systemet. Att utveckla icke-funktionella krav för att främja denna aspekt är något som vi kunnat gjort annorlunda, det är också ett bra tillvägagångssätt enligt Raturi \cite{raturi2014developing}. 

En förbättring som vi kunde gjort är att systemet ska ha ett icke-funktionellt krav att systemets beräkningar får ta en viss maximal tid. Detta är bra för energi- och miljösynpunkt eftersom hög processoranvändning under en lång tid leder till hög energiförbrukning för systemet. Det finns även en relation mellan hur systemets energianvändning påverkas av hårdvaran i systemet. Den största energianvändande komponenten är processorn och därför skulle systemet även kunna ha ett icke-funktionellt krav att optimera processoranvändningen, detta leder till energianvändningen för systemet minskar vilket är bra ur miljösynpunkt. Enligt Fan, Weber och Barroso \cite{fan2007power} ökar energiförbrukningen linjärt med processoranvändningen, detta är inte helt rättvisande och beror givetvis på vilken processor som används. 

För vårt system innebär det här att vi skulle vilja kombinera dessa två aspekter till ett gemensamt icke-funktionellt krav. En kombination av dessa innebär att vi som grupp skulle behövt göra en studie över hur vi kan optimera processoranvändningen med avseende på energiförbrukning samtidigt som systemet inte får för lång körtid. Det optimala ur energisparningssynpunkt kanske skulle vara att låta systemet använda 70\% av processorn och istället få lite längre körtid. Detta är någonting som vi skulle kunna undersökt i förstudiefasen för senare användning till kravprocessen.

För att bedöma de krav som vi har med hållbar utveckling i åtanke kommer endast de krav som inte har med kärnfunktionaliteten att göra bedömas, eftersom de är nödvändiga för systemets funktionalitet. Andra krav bedöms ur energisynpunkt, om koden är tillräckligt optimerad för att uppfylla kravet eller om det blir för tungt beräkningsmässigt som gör att kravet inte uppfylls. Det skulle även vara möjligt att sätta upp icke-funktionella krav med avseende på det sociala området som gör att användaren känner sig tillfredsställd med produkten och på så vis underhåller samhället. Detta bedöms genom att testa systemet med några användare och se huruvida det uppfyller användarens förväntningar eller inte. 

%%%%%%%%%%%%%%%%%%%%%%%%%%%%%%%%%%%%%%%%%%%%%%%%%%%%%%%%%%%%%%%%%%%%%%
%%% lorem.tex ends here
