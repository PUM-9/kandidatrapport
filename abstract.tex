Tekniken för att kunna skriva ut 3D-objekt har funnits i många år men först på 2010-talet har 3D-skrivare kommit ner i pris så pass att vanliga konsumenter kan köpa en.  Det finns dock ett problem: för att kunna använda 3D-skrivaren måste man antingen kunna CAD-mjukvara eller förlita sig på andra människors 3D-modeller. Genom att använda ett system för 3D-kopiering kan ett verkligt objekt istället kopieras.

Rapporten behandlar ett kandidatprojekt som genomfördes av sju studenter på datavetenskapliga civilingenjörsutbildningar på Linköpings universitet 2017. Rapporten beskriver utvecklingen av ett system för 3D-kopiering. Kunden i detta projekt var avdelningen för datorseende, institutionen för systemteknik vid Linköpings universitet. Målet med projektet var att utveckla ett system som kan ta separata punktmoln som indata för att sedan registrera dessa till ett komplett punktmoln. Utifrån detta kompletta punktmoln genereras sedan en 3D-mesh och utifrån den genereras G-code som kan föras över till en 3D-skrivare för att skriva ut objektet.
Tidigt i projektet var målet att vidareutveckla ett befintligt system. Detta mål omförhandlades med kunden efter att ett flertal fel identifierats med det befintliga systemet. 

Denna rapport undersöker också hur värde har skapats för kunden i projektet. Vilka erfarenheter som projektgruppen har kunnat dokumentera under projektets gång. Vilket stöd som projektgruppens systemanatomi har varit för projektet. Hur projektet påverkades av omförhandlingen med kunden och projektets tidiga mål, att vidareutveckla ett befintligt system, ändrade gruppens tillvägagångssätt.  

Projektet resulterade i 3DCopy, ett mjukvarusystem som registrerar punktmoln och utifrån dessa genererar 3D-mesh och G-code.

Rapporten innehåller sju bilagor med individuella bidrag där samtliga medlemmar genomfört en egen fördjupning eller undersökning. 

%%%%%%%%%%%%%%%%%%%%%%%%%%%%%%%%%%%%%%%%%%%%%%%%%%%%%%%%%%%%%%%%%%%%%%
%%% Abstract.tex ends here