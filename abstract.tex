\noindent
Tekniken för att kunna skriva ut 3D-objekt har funnits i många år men först på 2010-talet har 3D-skrivare blivit tillgängliga även för vanliga konsumenter. Det finns dock ett problem: för att kunna använda 3D-skrivaren måste man antingen kunna CAD-mjukvara eller förlita sig på andra människors 3D-modeller. Genom att använda ett system för 3D-kopiering kan ett verkligt objekt istället kopieras. Rapporten behandlar ett kandidatprojekt som genomfördes av sju studenter på datavetenskapliga civilingenjörsutbildningar på Linköpings universitet 2017. Målet med projektet var att utveckla ett system som kan ta separata punktmoln som indata för att sedan registrera dessa till ett komplett punktmoln. Utifrån detta kompletta punktmoln genereras sedan en 3D-mesh och utifrån den genereras G-code som kan föras över till en 3D-skrivare för att skriva ut objektet. Tidigt i projektet var målet att vidareutveckla ett befintligt system. Detta mål omförhandlades med kunden efter att ett flertal fel identifierats med det befintliga systemet. Projektet resulterade i \textit{3DCopy}, ett mjukvarusystem som registrerar punktmoln och utifrån dessa genererar en 3D-mesh.
\bigskip

\begin{center}
\textbf{Abstract}
\end{center}
\noindent

The technology to be able to print 3D objects has been available for many years, but it's only recently that 3D printers have been made available for regular consumers. There is one issue though: to be able to use the 3D printer either knowledge in CAD software or 3D models made by others are needed. By using a system for 3D copying a real object can instead be copied. This report presents a bachelor project that was done by seven students at engineering programs in computer science or software technology at Linköping University, 2017. The goal of the project was to develop a system that could take several point clouds as input and then register them to a complete point cloud. Then use this point cloud to generate a 3D mesh and then use it to generate G-code to be run on a 3D printer. The 3D printer will then be able to print the object. In the early stages of the project the main focus was to develop an already existing system. This goal was then renegotiated since the existing system contained several errors. The project resulted in \textit{3DCopy}, a software system that registers point clouds and from these point clouds generates a 3D mesh.

%%%%%%%%%%%%%%%%%%%%%%%%%%%%%%%%%%%%%%%%%%%%%%%%%%%%%%%%%%%%%%%%%%%%%%
%%% Abstract.tex ends here