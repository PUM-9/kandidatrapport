\chapter{Slutsatser}
\label{cha:conclusion}
Det inledande målet med projektet var att konstruera en mjukvara för att styra ett system som kopierar tredimensionella objekt. Målet reviderades efter halva utvecklingstiden eftersom det system som vi skulle bygga vidare på inte höll de förväntade kraven. Det nya målet med projektet blev istället att utveckla ett system som kan hantera, det vill säga registrera och mesha, punktmoln och på det viset försumma det tidigare projektets mjukvara. Målet med projektet nåddes, ett system för hantering av punktmoln har skapats. Det finns dock förbättringsmöjligheter med systemet. Att registrera punktmolnen så de bildar ett komplett punktmoln visade sig vara svårare än vad vi trott men vi har lyckats registrera ett visst antal punktmoln som gör systemet användbart. Det stora hindret med registreringen är en lång körtid, detta försvårade utvecklingsarbetet samt testningen.

Det finns även en del förbättringsmöjligheter angående meshningen. Den genererade meshen ifrån systemet är inte användbar för att skriva ut med hjälp av en 3D-skrivare. Meshen består av oönskad massa som inte är en del av det verkliga objektet. För att ta bort denna oönskade massa måste meshen hanteras manuellt. Detta problem är något som skulle kunna automatiseras så att det önskade objektet genereras direkt. Avsnitten nedan besvarar forskningsfrågorna som ställdes i avsnitt \ref{sec:research-questions}.

\section{Värde för kunden}

För att skapa värde för kunden med produkten är det framförallt två delar som är viktiga. För det första, att tidigt sammanställa de krav som kunden har på produkten så att arbetet tidigt går i rätt riktning. För det andra, att kontinuerligt följa upp dessa krav under projektets gång för att säkerställa att de inte har ändrats. Om dessa två steg följs kommer kunden att i slutändan få en produkt som uppfyller kundens krav.

Något som också tagits hänsyn till i utvecklingen av systemet är dess ursprungliga syfte, att vara del av ett större system. Därför har det fokuserats på att göra 3DCopy modulärt för att det ska vara enkelt för kunden att anlita en annan grupp med liknande programmeringsvana som kan integrera 3DCopy i ett större system, vidareutveckla 3DCopy eller använda 3DCopy för att forska kring registrering och meshgenerering.

\section{Erfarenheter}

Att förbereda sig är en avgörande del i ett projekt. Under projektets gång har vi insett att man behöver pröva nya områden inte bara i teorin utan även med praktiska övningar. Vi har också insett att man inte kan undersöka en enda sak allt för mycket utan bör sprida ut sin tid och undersöka flera områden. Detta gäller lika mycket för ramverk, språk och teori som för de system man bygger vidare på eller som ligger till grund för det man ska utveckla. Man kan inte lita på att det är som det verkar utan bör alltid ha god dokumentation för sina antaganden.

Det är viktigt att ha strikta kommunikationskanaler och att utnyttja de tjänster och program som finns är till stor hjälp under ett projekt. Kunskaper i \textit{Slack} kommer att komma till bra användning i våra fortsatta arbetsliv, precis som att arbeta med aktiviteter och bryta ner problem i så små bitar som möjligt.

Granskning av kod och dokument är också något som har spelat stor roll i detta projekt och kan appliceras på framtida projekt. Granskningsprocessen är ett utmärkt verktyg då den bidrar med framförallt två viktiga aspekter. För det första kontrolleras allt gruppen producerar och kvalitén på det producerade håller därför hög standard. För det andra hjälper den till att sprida kunskap mellan gruppens medlemmar då man tar del av det andra gruppmedlemmar har gjort och lär sig av varandra.

Dessa erfarenheter är av intresse för framtida projekt då 3DCopys projektgrupp har fått stor hjälp av väl genomförda efterforskningar, tydlig kommunikation och noggranna granskningar av dokument. För framtida projekt innebär det en ökad effektivitet om de tar del av och applicerar dessa erfarenheter.

\section{Systemanatomi}
För att svara på fråga 3 i avsnitt \ref{sec:research-questions} så var projektets systemanatomi användbar vid projektets början. Systemanatomin gav ett bra stöd under projektets utvecklande eftersom alla i projektgruppen fick en enkel helhetsbild över vilken funktionalitet systemet skulle innehålla. Projektgruppen skulle bygga ett stort och komplext system och här var systemanatomin till en stor hjälp för projektgruppen då den gav en bättre förståelse för produkten och dess olika funktioner. Systemanatomin gav oss alltså en gemensam förståelse över produkten samt över funktionerna och dess beroenden vilket ledde till att alla i projektmedlemmar var på samma nivå kunskapsmässigt vid projektets start.

Systemanatomin var även till stor hjälp då arkitekturen togs fram då den tydliggjorde vilka funktioner som systemet skulle innehålla och hur närliggande funktionalitet kunde delas in i relevanta moduler. På så sätt hjälpte systemanatomin till att göra systemet så modulärt som möjligt vilket i slutändan ledde till en högre kvalitet på produkten.

\section{Reviderad kravspecifikation}
För att svara på fråga 4 i avsnitt \ref{sec:research-questions} så ändrades kort sagt hela projektet. Efter omförhandling med kunden och samtal med handledaren och examinatorn ändrades målet med projektet tillsammans med kraven. Under omförhandlingen med kunden kom vi överens att det nya målet var likt det gamla fast vi kopplade bort TreeD, och tillsammans med det, tog bort eller skrev om de kraven som direkt eller indirekt byggde på TreeDs mjukvara. Som tur var så hade vi kod som kunde återanvändas efter denna pivotering av projektet. Det är denna kod som det nya systemet bygger på. Det som förvärrade situationen var att omförhandlingen kom vid helt fel tidpunkt eftersom planeringen av den andra halvan av projektet inkluderade mycket mer dokumentskrivande än de tidigare iterationerna.

Utöver detta så påverkades gruppen. Precis innan omförhandlingen så var stämningen i gruppen relativt dålig. Det kännes som att vi tog ett steg framåt och två bakåt nästan varje dag. Gruppens stämning och sammanhållning blev gradvis bättre efter omförhandlingen, men problemen med projektet var som ett mörkt moln som hängde över gruppen.

Det finns inget lätt sätt att hantera en sådan kraftig omförhandling av ett projekts krav och mål som skedde i projektet. Det bästa sättet att handskas med en sådan abrupt pivotering är att kollektivt ta ett djupt andetag, samla nya krafter och jobba på.

\section{Vidareutveckling av ett system}
% Hur har gruppens tillvägagångssätt ändrats på grund av faktumet att vi vidareutvecklade ett system? 
För att svara på fråga 5 i avsnitt \ref{sec:research-questions} så ändrades tillvägagångssättet för arbetet markant. Då vi skulle utveckla ett system som skulle vara tätt integrerat med det tidigare systemet som använde ROS lade vi mycket fokus i början på att lära oss ROS. Hela arkitekturen för vårt system togs fram med ROS i åtanke.

Mycket arbete lades även på att utforska det tidigare systemet och förstå hur det fungerar. Exempelvis hade de punktmoln som genererades en del skräpdata som krävde en hel del arbete för att kunna filtrera bort på ett stabilt och konsekvent sätt.

Gruppens tillvägagångssätt påverkades även på så sätt att när det tidigare systemet inte klarade av att användas som det var planerat var gruppen tvungna att omförhandla kravspecifikationen sent i projektet. Detta resulterade i att en ny arkitektur snabbt behövde tas fram och gruppen behövde snabbt anpassa sig till att utveckla ett nytt program med en lägre koppling till det tidigare systemet.

För att sammanfatta innebar det tidigare systemet att vi kunde utveckla saker som vi inte hade kunnat göra utan det eftersom det var mycket arbete som vi inte behövde göra. Däremot var det inte helt problemfritt och mycket tid och energi har lagts på att utforska och lära oss systemet samt att lösa problem som uppstått med det tidigare systemet.

% I vilken utsträckning har målet uppnåtts?
% Vad är det svaren på forskningsfrågorna?

% Konsekvenserna för målgruppen (och möjligen för forskare och utövare) måste också beskrivas.
% Det borde finnas en sektion om framtida arbete där idéer för fortsatt arbete beskrivs.
% Om slutsats kapitlet innehåller ett sådant avsnitt så måste idéerna vara konkreta och väl genomtänkta.

%%%%%%%%%%%%%%%%%%%%%%%%%%%%%%%%%%%%%%%%%%%%%%%%%%%%%%%%%%%%%%%%%%%%%%
%%% lorem.tex ends here