\chapter{Slutsatser}
\label{cha:conclusion}

\subsection{Systemanatomi}
Projektets systemanatomin var användbar innan revideringen av projektet. Det gav ett bra stöd under projektets utvecklande eftersom alla i projektgruppen fick en enkel helhetsbild över hur systemet var tänkt att byggas. Systemanatomin gav en gemensam förståelse över produkten samt över funktionerna och dess beroenden vilket ledde till att alla i projektmedlemmar var på samma nivå vid projektets start.

% En sammanfattning av syftet och forskningsfrågorna.
% I vilken utsträckning har målet uppnåtts?
% Vad är det svaren på forskningsfrågorna?

% Konsekvenserna för målgruppen (och möjligen för forskare och utövare) måste också beskrivas.
% Det borde finnas en sektion om framtida arbete där idéer för fortsatt arbete beskrivs.
% Om slutsats kapitlet innehåller ett sådant avsnitt så måste idéerna vara konkreta och väl genomtänkta.


%%%%%%%%%%%%%%%%%%%%%%%%%%%%%%%%%%%%%%%%%%%%%%%%%%%%%%%%%%%%%%%%%%%%%%
%%% lorem.tex ends here