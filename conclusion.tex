\chapter{Slutsatser}
\label{cha:conclusion}
Det inledande målet med projektet var att konstruera en mjukvara för att styra ett system som kopierar tredimensionella objekt. Målet reviderades efter halva utvecklingstiden eftersom det system som vi skulle bygga vidare på inte höll de förväntade kraven. Det nya målet med projektet blev istället att utveckla ett system som kan hantera (registrera och mesha) punktmoln och på det viset försumma det tidigare projektets mjukvara. Målet med projektet nåddes, ett system för hantering av punktmoln har skapats. Det finns dock förbättringsmöjligheter med systemet. Att registrera punktmolnen så de bildar ett komplett punktmoln visade sig vara svårare än vad vi trott men vi har lyckats registrera ett visst antal punktmoln som gör systemet användbart. Det stora hindret med registreringen var en lång körtid, detta försvårade utvecklingsarbetet samt testningen. 

Det finns även en del förbättringsmöjligheter angående meshningen. Den genererade meshen ifrån systemet är inte användbar för att skriva ut med hjälp av en 3D-skrivare. Meshen består av oönskad massa som inte är en del av det verkliga objektet. För att ta bort denna oönskade massa måste meshen hanteras manuellt. Detta problem är något som skulle kunna automatiseras så att det önskade objektet genereras direkt. Avsnitten nedan besvarar på forskningsfrågorna som ställdes i avsnitt \ref{sec:research-questions}

\section{Värde för kunden}
\section{Erfarenheter}
\section{Systemanatomi}
För att svara på fråga 3 i \ref{sec:research-questions} så var projektets systemanatomin användbar innan revideringen av projektet. Det gav ett bra stöd under projektets utvecklande eftersom alla i projektgruppen fick en enkel helhetsbild över hur systemet var tänkt att byggas. Systemanatomin gav en gemensam förståelse över produkten samt över funktionerna och dess beroenden vilket ledde till att alla i projektmedlemmar var på samma nivå kunskapsmässigt vid projektets start.

\section{Reviderad kravspecifikation}
\section{Vidareutveckling av ett system}

% En sammanfattning av syftet och forskningsfrågorna.
% I vilken utsträckning har målet uppnåtts?
% Vad är det svaren på forskningsfrågorna?

% Konsekvenserna för målgruppen (och möjligen för forskare och utövare) måste också beskrivas.
% Det borde finnas en sektion om framtida arbete där idéer för fortsatt arbete beskrivs.
% Om slutsats kapitlet innehåller ett sådant avsnitt så måste idéerna vara konkreta och väl genomtänkta.

%%%%%%%%%%%%%%%%%%%%%%%%%%%%%%%%%%%%%%%%%%%%%%%%%%%%%%%%%%%%%%%%%%%%%%
%%% lorem.tex ends here