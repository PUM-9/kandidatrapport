\chapter{Hur kravhanteringsmetoder påverkar ett utvecklingsprojekt}
\label{cha:indiv-report-jannering}
\chapterprecis{\LARGE{---- Gustav Jannering ----}}

\section{Inledning}
\label{sec:introduction-jannering}

Denna del detaljerar projektets analysansvarige Gustav Jannerings utredning kring kravhantering och kravinsamling

\subsection{Syfte}
\label{sec:purpose-jannering}


Syftet med denna rapport är att undersöka hur olika metoder för kravhantering och elicitering av krav påverkar ett utvecklingsprojekt av samma storlek som det projekt som övriga rapporten beskriver (ett projekt med 7-8 utvecklare med en budget på 400 timmar vardera). Samt vilka för och nackdelar som finns med den metoden som användes och vad som kunde gjorts annorlunda.

\subsection{Frågeställning}
\label{sec:issue-jannering}

\subsubsection{Generella frågeställningar}
\begin{enumerate}
	\item Hur kan IEEE std 830 användas i ett programvaruutvecklingsprojekt för att underlätta kravhantering och elicitering av krav?
	
	\item Vilka metoder finns för kravhantering och elicitering av krav och vilka fördelar och nackdelar finns med dessa? 
\end{enumerate}
\subsubsection{Specifika frågeställningar}
\begin{enumerate}
	\item Vilka fördelar och nackdelar finns med att använda IEEE std 830 i ett programvaruutvecklings projekt av den storleken som detta projekt?
	
	\item Vilka erfarenheter kan dokumenteras från kravhantering och elicitering av krav i projektet som kan vara intressanta
	för framtida projekt?
	
\end{enumerate}
\subsection{Definitioner, akronym och förkortningar}
Följande definitioner och förkortningar används på flera ställen i denna del av rapporten:
\begin{itemize}
	\item elicitering av krav - Identifiering och uppsamling av krav, översatt från det engelska ordet ”requirements elicitation”
	\item kravhantering - systematisk arbete med behov som ska uppfyllas av tekniska system. Översatt från det engelska ordet "requirements engineering"
\end{itemize}
\section{Bakgrund}
\label{sec:background-jannering}

%% Skriv här

\section{Teori}
\label{sec:theory-jannering}

%% Skriv här

\section{Metod}
\label{sec:method-jannering}

För att undersöka vilka metoder som används för kravhantering har en litteraturstudie genomförts för att identifiera vilka metoder som är populära och hur dessa används. Efter att etablerade metoder för kravhantering och elicitering av krav identifierats så genomfördes ett par experiment med övriga gruppmedlemmar för att undersöka en given metods fördelar och nackdelar. Efter experimenten så utvärderade gruppen metoden som användes för att identifiera för- och nackdelar. 

För att identifiera för- och nackdelar med IEEE std 830 [1] så har dels gruppen och dels analysansvariga från andra projektgrupper i denna kurs svarat på en kort undersökning.


\section{Resultat}
\label{sec:results-jannering}

%% Skriv här

\section{Diskussion}
\label{sec:discussion-jannering}

%% Skriv här

\section{Slutsatser}
\label{sec:conclusions-jannering}

%% Skriv här

\section{Tänkta referenser}
[1]  "IEEE recommended practice for software requirements specifications - IEEE Xplore document," 2017. [Online]. Available: http://ieeexplore.ieee.org/servlet/opac?punumber=5841. Accessed: Mar. 6, 2017.

%%%%%%%%%%%%%%%%%%%%%%%%%%%%%%%%%%%%%%%%%%%%%%%%%%%%%%%%%%%%%%%%%%%%%%
%%% Jannering-report.tex ends here
