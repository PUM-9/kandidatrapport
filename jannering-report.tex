\chapter{Hur kravhanteringsmetoder påverkar ett utvecklingsprojekt}
\label{cha:indiv-report-jannering}
\chapterprecis{\LARGE{---- Gustav Jannering ----}}

\section{Inledning}
\label{sec:introduction-jannering}

Denna del detaljerar projektets analysansvarige Gustav Jannerings utredning kring kravhantering och kravinsamling

\subsection{Syfte}
\label{sec:purpose-jannering}


Syftet med denna rapport är att undersöka hur olika metoder för kravhantering och elicitering av krav påverkar ett utvecklingsprojekt av samma storlek som det projekt som övriga rapporten beskriver (ett projekt med 7-8 utvecklare med en budget på 400 timmar vardera). Samt vilka för och nackdelar som finns med den metoden som användes och vad som kunde gjorts annorlunda.

\subsection{Frågeställning}
\label{sec:issue-jannering}

\subsubsection{Generella frågeställningar}
\begin{enumerate}
	\item Vilka metoder finns för kravhantering och elicitering av krav och vilka fördelar och nackdelar finns med dessa? 
\end{enumerate}
\subsubsection{Specifika frågeställningar}
\begin{enumerate}
	\item Vilka fördelar och nackdelar finns med att använda IEEE std 830 i ett programvaruutvecklings projekt av den storleken som detta projekt?
	
	\item Vilka erfarenheter kan dokumenteras från arbetet med krav i projektet som kan vara intressanta för framtida projekt?
	
\end{enumerate}
\subsection{Avgränsningar}
Denna rapport kommer att begränsas till de frågeställningar som presenterats ovan. För metoder för elicitering av krav har följande avgränsning gjorts: endast generella metoder som har studerats vetenskapligt har studerats. Inga andra metoder (icke generella eller metoder som inte studerats vetenskapligt) kommer att presenteras i denna rapport. De erfarenheter som dokumenteras senare in denna rapport kommer uteslutande från kursen \textbf{TDDD96 	Kandidatprojekt i programvaruutveckling} och mer specifikt från det projekt i kursen som presenterats tidigare i denna rapport.   
\subsection{Definitioner, akronym och förkortningar}
Följande definitioner och förkortningar används på flera ställen i denna del av rapporten:
\begin{itemize}
	\item Elicitering av krav - Identifiering och uppsamling av krav, översatt från det engelska ordet ”requirements elicitation”
	\item Kravhantering - systematisk arbete med behov som ska uppfyllas av tekniska system. Översatt från det engelska ordet "requirements engineering"
	\item IEEE - Institute of Electrical and Electronics Engineers, en icke-vinstorienterad organisation bestående av ingenjörer och vetenskapspersoner. Upprätthåller standarder inom ingenjörsvetenskap.
	\item Gamification - Processen att lägga till spel eller spellika element till något (som en uppgift) för att uppmuntra deltagande
	\item User story - En informell, naturlig språkbeskrivning av en eller flera funktioner i ett mjukvarusystem.
	\item Acceptance test - Ett test utfört för att avgöra om kraven i en specifikation eller ett kontrakt är uppfyllda.
\end{itemize}
\section{Bakgrund}
\label{sec:background-jannering}

Arbetet med kravhantering och elicitering av krav i projektet (som presenteras tidigare i denna rapport) började vid det första mötet med kunden. Detta var ett möte, dels för att träffa kunden efter att projektgruppen hade blivit tilldelade projektet, och dels för att starta kravinsamlingsfasen. Kunden i projektet, CVL, sade vid detta möte att de helst ville styra projektet med en lös hand och att det var mer eller mindre upp till projektgruppen att komma med en lista på förslag till krav (i form av ett första utkast till en kravspecifikation). För att skapa denna lista med förslag till krav på den slutgiltiga produkten så gick gruppen gemensamt igenom det projektdirektiv som CVL tidigare hade presenterat (återfinns i Bilaga \ref{appendix:projekt_direktiv}) för att identifiera vilka funktioner som produkten skulle ha. Utifrån dessa funktioner kom gruppen med förslag på krav. Denna metod kallas i facklitteratur för "introspektion" \cite{goguen1993techniques}. Användandet av denna metod innebar att kunden bara var involverad i att godkänna de krav som gruppen föreslog och var minimalt involverade i kravidentifieringsprocessen. Denna metod kan också leda till att kund och utvecklingsgruppen har olika uppfattningar om hur den slutgiltiga produkten ska se ut. 

Examinatorn i kursen (TDDD96 Kandidatprojekt i programvaruutveckling, Linköpings universitet), som detta projekt genomfördes under, hade bestämt att IEEE std 830 var den standard som kravspecifikationen skulle följa (för mer information om IEEE std 830 se kap C.3). Som kan ses i kap C.3 är denna standard väldigt utförlig, och behöver i många fall skräddarsys för det specifika projektet. Detta projekt var första gången då någon i projektgruppen använde standarden. Det medförde att vi inte hade tillräcklig kunskap för att skräddarsy standarden till projektet, istället så följde vi standarden så gott vi kunde. Målet med en kravspecifikation är att identifiera kundens vision på hur den slutgiltiga produkten ska se ut. Se över om detta faktiskt är det som kunden vill ha eller behöver och sedan specificera de krav som produkten ska vara bunden av. 


\section{Teori}
\label{sec:theory-jannering}
Kravhantering kan definieras som ” systematisk arbete med behov som ska uppfyllas av tekniska system”. Detta arbete ligger ofta tidigt i ett projekt, då det är viktigt att specificera de krav som skall uppfyllas. Kravhanteringsprocessen är ofta uppdelad i fyra delar: elicitering, analys, specifikation och godkännande. Detta arbete tar främst upp de tre första delarna. I detta arbetes syfte så definieras elicitering som: ”insamlingen av krav eller information för senare specificering av krav från användare, kunder och andra intressenter”. I analysfasen analyseras de kraven och den informationen som samlades in under eliciterinsfasen för att undersöka huruvida dessa krav är relevanta och huruvida informationen går att använda för specificera krav. Under specificeringsfasen så ska de slutgiltiga kraven specificeras. Detta innebär att varje krav numreras och att en kort text skrivs som detaljerar kravet.
\subsection{IEEE standard 830}
IEEE standard 830 är en standard, publicerad av IEEE, som specificerar innehållet och kvaliteterna hos en ”bra” kravspecifikation.\cite{ieee1998ieee} Standarden specificerar både hur en kravspecifikation borde skrivas (hur involverad kund/intressenter ska vara, hur kravspecifikationen borde utvecklas med tiden m.m.) och vilka avsnitt som borde finnas med och deras syften. Kraven som skapas med standarden ska vara: Korrekta (eller relevanta), Entydiga, Kompletta, Konsekventa, Klassificerade för betydelse, Verifierbara, Modifierbara, Spårbara.

\section{Metod}
\label{sec:method-jannering}

För att undersöka vilka metoder som används för kravhantering har en litteraturstudie av publicerade vetenskapliga rapporter genomförts för att identifiera vilka metoder som har studerats och hur dessa används idag. Svårigheten med denna ansats har varit att begränsa antalet källor och försöka göra relevanta avgränsningar för att passa arbetes begränsningar. De rapporter som har studerats har uteslutande fokuserat på eliciteringsprocessen av kravhantering. Vidare har endast de artiklar som presenterat eller studerat generella metoder använts. Dessa avgränsningar är gjorda för att göra denna rapport relevant för både mindre och större utvecklingsprojekt.
  
För att identifiera för- och nackdelar med IEEE std 830 \cite{ieee1998ieee} så har dels gruppen och dels analysansvariga från andra projektgrupper i denna kurs intervjuats i en öppen intervju Se Bilaga \ref{appendix:intervju_guide} för den intervju guide som användes vid intervjuerna

För att dokumentera erfarenheter från arbetet med kravhantering och elicitering av krav så gick jag tillsammans med övriga projektgruppen gått igenom den kravspecifikation som framställdes i projektet och utvärderade vårt arbete. Efter detta tillfälle har jag också själv dokumenterat erfarenheter under mitt arbete med denna rapport.




\section{Resultat}
\label{sec:results-jannering}
\subsection{Metoder för elicitering av krav}
På en vetenskaplig konferens i San Diego 1993 presenterade C. Linde och J. A. Goguen en rapport som detaljerade författarnas undersökning och utvärdering av olika metoder för elicitera krav \cite{goguen1993techniques}. I rapporten detaljerar författarna följande metoder: Introspektion, öppna intervjuer, frågeformulär och fokusgrupper. I det följande hänvisas enbart till C. Linde och J. A. Goguen rapport. \cite{goguen1993techniques}

\subsubsection{Introspektion}
Introspektion kallas den metod som användes i projektarbetet som presenterats tidigare i denna rapport. För att elicitera krav med introspektion så föreställer sig personen som eliciterar krav vilket system som hen skulle vilja ha om hen var kunden. Detta är den mest uppenbara metoden för att specificera krav men  C. Linde och J. A. Goguen skriver i sin rapport att det finns fall då introspektion misslyckas med att specificera viktiga krav. Detta kan ofta bero på personenliga fördomar (från eng. bias). Två personer med olika bakgrund (både personlig och professionell) kan komma med olika krav då de använder introspektion för att elicitera krav. Det är även svårt för en enskilld person att identifiera samtliga krav som bör ställas på ett system.  C. Linde och J. A. Goguens slutsats är att introspektion är en duglig metod för elicitering av krav om den kombineras med andra metoder och att de eliciterar krav med hjälp av introspektion kommer från olika bakgrunder. 

\subsubsection{Frågeformulär}
Frågeformulärsintervjuer används inom många forskningsområden. Formulär där försökspersoner får svara på frågor med förbestämda svar är ett bra sätt för forskare att få statistisk information. Detta tillvägagångssätt kan anpassas för att användas för att elicitera och specificera krav. Tänkta användare, kunder och andra intressenter får, med hjälp av den som eliciterar krav, fylla i ett formulär, vars data senare analyseras för att identifiera vad som är relevant för de som svarade på formuläret. Liksom i forskningsvärlden så är detta ett sätt att få statistik information som kan visas för kunden och användas som underlag i en mängd situationer under utvecklingen av ett system. Författarna pointerar att det finns ett fundamentalt problem med detta tillvägagångssätt, Vad händer om den som skapar formuläret och frågorna och den som svarar på frågorna inte har samma referensram?
 
\subsubsection{Öppna intervjuer} 
 C. Linde och J. A. Goguen kommer fram till att ett sätt att lindra problemet med frågeformulärsmetoden är att istället hålla s.k. öppna intervjuer. Intervjuaren ställer en fråga och låter försökspersonen svara på frågan med egna ord. Intervjun fortskrider med ett antal förskriva frågor som ska ställas och besvaras men intervjuaren kan ställa följdfrågor och egna frågor emellan de förskriva frågorna. Denna metod används ofta av psykologer, antropologer och andra som forskare inom liknande områden. För att anpassa denna metod för elicitering av krav så ändrar man vilka frågor man ställer och vilka personer man ställer frågorna till. Frågorna kommer att handla om hur försökspersonen använder system som liknar det system som ska utvecklas. Det är viktigt att frågorna är konstruerade så att de inte leder till att försökspersonen använder introspektion för att själv föreställa vilka krav som personen tycker systemet bör ha. Försökspersonerna kommer, liksom med frågeformulärsmetoden, att vara tänkta användare, kunder och andra intressenter. 

\subsubsection{Fokusgrupper}
Den sista metoden för elicitering av krav som C. Linde och J. A. Goguen presenterar i sin rapport är fokusgrupper (från eng. focus groups). Fokusgrupper är mycket vanligt i marknadsundersökningar. Tanken bakom fokusgrupper är att samla en grupp intressenter (potentiella kunder om en marknadsundersökning utförs) med olika bakgrunder. Gruppen får sen diskutera ett ämne som är av intresse (t.ex. en ny produkt om en marknadsundersökning utförs). Arrangören för protokoll på vad som sägs i gruppen och använder senare denna data som underlag.  För att applicera denna metod för att elicitera krav så skapas JAD (”Joint Application Development”) eller RAD (”Rapid Application Development,”) grupper bestående av utvecklare kunder och andra intressenter som i grupp diskuterar det system som ska utvecklas. Författarna pointerar att det är svårt att inkludera personer som saknar teknisk bakgrund i dessa samtal då de ha svårigheter att bedöma betydelsen av tekniska beslut. Författarnas slutsats är att denna metod är lovande men att dess begränsningar bör studeras. 

\subsubsection{GREM}
I en rapport som presenterades på International Working Conference on Requirements Engineering: Foundation for Software Quality I Göteborg 2016 skriver P. Lombriser et al. om en ny metod för elicitering av krav som bygger på gamification \cite{lombriser2016gamified}. Rapporten beskriver metod som ska öka engagemanget hos interessenter vid elicitering av krav som författarna kallar ”gamified requirements engineering model” eller ”GREM”. I rapporten presenterar författarna ett experiment där en webbaserad plattform, som har spellika element, använts för att elicitera krav genom att skapa user stories och acceptence tests. Författarnas slutsats är att deras experiment visar att användandet av spellika element kan ha en positiv inverkan på intressenters engagemang men att denna inverkan kan variera baserat på vilka spellika element som används.

\subsubsection{Kravkategorisering}
I sin doktorsavhandling definierar M. Karlsson tre olika kategorier för krav: fångade krav (från eng. captured requirements), eliciterade krav (från eng. elicited requirements) och framväxande krav (från eng. emerget requirements).\cite{lkp.26083619960101} Fångade krav är krav som är relativt lätta att specificera eller som är relativt uppenbara. Eliciterade krav är krav som identifierats efter att grävt djupare i vad som tänkta användare, kunder och andra intressenter förväntar sig av systemet. Dessa krav kan klassas som icke uppenbara och kräver en grad av utredning för att identifieras. Framväxande krav är krav som växer fram under utvecklingen av systemet allt efter att tänkta användare, kunder och andra intressenter fått testa systemet. Dessa krav är svåra att identifiera i ett utvecklingsprojekts tidiga stadier om utvecklingsgruppen inte håller prototypdemonstationer där tänkta användare får komma med feedback. 
 
\subsection{Erfarenheter}
Följande är en sammanfattning av de erfarenheter om kravhantering och elicitering av krav som dokumenterats under projektets gång.
\begin{itemize}
	\item Att skapa en ”bra” kravspecifikation är svårt
	\item För att identifiera relevanta krav gäller det att både utvecklingsgruppen och kunden är tillräckligt insatta i det som ska utvecklas. (speciellt om projektets mål är att vidareutveckla ett system.
	\item Att blint följa IEEE std 830 för att skriva en kravspecifikation kan vara frustrerande.
	\item Kunden (och/eller tänkta användare) borde vara mer involverad vid specificering av krav än de var i projektet.
	\item Vid vidareutveckling är det viktigt att testa det befintliga systemet ordentligt innan man specificerar krav.
\end{itemize}

\subsection{IEEE standard 830}
Följande är en sammanfattning av två öppna intervjuer gjorda med: Oskar Magnusson (analysansvarig i PUM grupp 8 2017) och Petter Granli (analysansvarig i PUM grupp 5 2017). Intervju guiden som användes vid dessa intervjuer återfinns i Bilaga \ref{appendix:intervju_guide}. Svaren på intervjuerna är anonymiserade.  

Både intervjuobjekten använde introspektion för att elicitera krav i sina projekt. De började med att analysera det projektdirektiv som gavs för respektive projekt och eliciterade krav utifrån den och skapade ett första utkast till en kravspecifikation som de sedan skickade till sina respektive kunder. När det gäller användandet av IEEE std 830 så svarade det ena intervjuobjektet att de inte använt sig av denna standard vid skrivandet av kravspecifikationen. Intervjuobjektets grupp baserade sin kravspecifikation på en mall från en tidigare projektkurs och undersökte IEEE std 830 först efter att en opponerande grupp hade påpekat att IEEE std 830 var standarden som skulle användas. Det andra intervjuobjektets grupp använde sig av IEEE std 830 från början. Efter utfrågning om vad de tyckte om standarden höll båda intervjuobjekten med om att standarden gav en bra mall att följa när det gällde vilka rubriker som bör vara med i en kravspecifikation men att standarden kändes för omständlig för ett projekt av denna omfattning och att de kände att det tog för lång tid att läsa igenom och sätta sig in i standarden.          

\section{Diskussion}
\label{sec:discussion-jannering}

\subsection{För- och nackdelar med metoder för elicitering av krav}
Nedan används M. Karlssons kravkategorisering för att diskutera vilka metoder som är kapabla att specificera vilka krav. Den tredje kategorin av krav, ”framväxande krav”, är mycket svåra att korrekt identifiera tidigt i ett projekt så den kommer inte att användas.
\subsubsection{Introspektion}
Min åsikt om för- och nackdelarna med introspektion stämmer överens med C. Lindes och J. A. Goguens slutsats om introspektion\cite{goguen1993techniques}. Introspektion är en duglig metod för elicitering av krav om den kombineras med andra metoder och att de eliciterar krav med hjälp av introspektion kommer från olika bakgrunder. Ren introspektion kräver mycket erfarenhet för att kunna elicitera alla relevanta krav som bör ställas på ett system. Det finns också en risk att någon som är har erfarenhet missar relevanta krav för att personen förlitar sig på sin erfarenhet för mycket. Introspektion klarar i nästan alla fall att identifiera alla ”fångade krav”, eftersom de kraven är mer eller mindre uppenbara. Men det krävs en betydande mängd erfarenhet för att identifiera samtliga ”eliciterade krav”.

\subsubsection{Frågeformulär}
Detta tillvägagångsätt, liksom introspektion, är beroende av att den som eliciterar krav (och i detta fall skapar frågorna och svaren) har en korrekt bild av vad kunden, tänkta användare och andra intressenter vill ha för system. Om denna bild inte är korrekt så finns det en stor risk att de krav som eliciteras inte är relevanta. Liksom introspektion så klarar frågeformulär av att elicitera ”fångade krav”, men om den som skapar frågorna och svaren inte och försökspersonen inte delar samma referensram så finns risken att vissa ”eliciterade krav” missas 

\subsubsection{Öppna intervjuer}
Även om denna metod lindrar problemet med frågeformulärsmetoden så finns detta problem kvar. Om intervjuaren och försökspersonen inte har samma uppfattning om vad som ska utvecklas eller vilka systemets fundamentala aspekter är så finns risken att kraven som specificeras med hjälp av intervjun inte är relevanta. Öppna intervjuer kommer nästan alltid att identifiera alla ”fångade krav”, men liksom frågeformulär finns det en risk (som är mindre i detta fall) att vissa ”eliciterade krav” inte specificeras. Ett annat problem med denna metod är tidsaspekten. Öppna intervjuer tar tid. Eftersom intervjuerna görs en och en kommer tiden det tar att elicitera krav från samma mängd intressenter att ta mycket längre tid om man använder öppna intervjuer istället för frågeformulär.  

\subsubsection{Fokusgrupper}
En risk med fokusgrupper är att deltagarna i gruppen inte representerar verkligheten, d.v.s. att deltagarna inte lyckas representera en eller flera grupper av intressenter, vilket leder till att vissa relevanta krav inte identifieras. Det finns också andra risker med fokusgrupper. Eftersom en fokusgrupp ska starta en diskussion kring ett ämne så finns risken att en minoritet av gruppen inte blir hörda eller att de hörs för mycket. Det kan vara lätt för en majoritet att utesluta en minoritet från diskussionen eller att en minoritet tar över diskussionen. Detta kan leda till att de som representerar en grupp intressenter inte blir hörda och att relevanta krav inte blir eliciterade. Fokusgrupper har enligt min uppfattning störst chans att korrekt identifiera samtliga ”eliciterade krav” så länge gruppen fungerar korrekt.

\subsubsection{GREM}
GREM verkar vara en lovande metod men den är väldigt ung (rapporten som presenterade metoden skrev 2016) och är beroende av extern mjukvara. GREM har potential men bör studeras mer och/eller standardiseras innan den används mer brett. I teorin har GREM potentialen att korrekt identifiera alla ”eliciterade krav” i from av user stories och acceptance tests skapade av kunder, eftersom GREMs mål är att involvera kunder, användare och andra intressenter i eliciteringsprocessen.

\subsection{För- och nackdelar med IEEE std 830}
Att intervjua två personer för att identifiera för- och nackdelar med en standard som är skapad av en organisation med tusentals medlemmar, och mycket större kunskap och erfarenhet av ämnet, är inte optimalt. Men det kan ge en bra bild av ur amatörer som försöker sätta sig in i standarden känner. efter intervjuerna och efter att reflekterat på mina egna tankar om standarden kan jag konstatera att våra åsikter om standarden är väldigt lika. IEEE std 830 ger en bas att bygga en "bra" kravspecifikation på men i detta fall, ett mindre projekt med relativt få projektmedlemmar, så känns standarden överflödig.   
\subsection{Erfarenheter}
Erfarenheterna som finns listade i kap. C.5.2 är specifika för detta projekt, ett projekt där kraven fick skrivas om efter att fel med den befintliga mjukvaran som skulle vidareutvecklas hittats. Detta är ett tveeggat svärd, dels ger det bra erfarenheter att se tillbaka på om hur vi borde ha gjort för att identifiera dessa fel tidigare. Men det som (i min egna åsikt) ledde till att vi inte identifierade felet i ett tidigt skede är en kedja med val (både från vår och från kundens sida) som verkade försumbara, men som i efterhand visar att vi, från gruppens sida, borde ha ställt mer krav på kundens kunskap av systemet innan vi specificerade krav på vår vidareutveckling av den befintliga mjukvaran som implicit antog att den befintliga mjukvaran var relativt felfri. med detta i åtanke skulle jag hävda att de erfarenheter som är listande i kap. C.5.2 alla är relevanta för framtida projekt.      
\subsection{Källkritik}
De källor som använts i denna rapport är alla publicerade källor som citerats i tidigare arbeten. Den första källan som har använts är publicerad av IEEE. som nämns i kap. C.1.3 är IEEE en icke-vinstdrivande organisation som upprätthåller standarder inom ingenjörsvetenskap. Den andra källan är publicerad av ett stort förlag och presenterades dessutom på en internationell konferens. Den tredje och sista är en doktorsavhandling från Chalmers. Jag skulle hävda att alla dessa källor är pålitliga.      
\subsection{Alternativa metoder}
Det finns en uppsjö  med metoder som skulle kunna använts för att besvara de frågor som ställs i kap. C.1.2. Problemet i detta fall har varit att finna metoder för att besvara frågorna som passade detta arbetes tidsbegränsningar. Av denna anledning valdes just en begränsad litteraturstudie tillsammans med öppna intervjuer och en utvärdering för att besvara frågorna. Om tidsbegränsningen för detta arbete varit annorlunda kunde man utvidgat både litteraturstudien och de öppna intervjuerna.   
\section{Slutsatser}
\label{sec:conclusions-jannering}
\subsection{Vilka metoder finns för kravhantering och elicitering av krav och vilka fördelar och nackdelar finns med dessa?}
De metoder för elicitering av krav som presenterats är: introspektion, Frågeformulär, öppna intervjuer, fokusgrupper och GREM.

\subsubsection{Introspektion}
Introspektion är den vanligaste metoden för att eliceitera krav men min slutsats är att i många fall så är detta inte den bästa metoden att använda. Det finns för många felkällor med denna metod för att säga att man borde använda "ren introspektion" (bara introspektion) för att elicitera krav. Denna metod bygger på att den som eliciterar krav lyckas få en komplett bild av det system som kunden/användaren vill ha. Denna bild kommer att vara komplex, även i mindre projekt, vilket betyder att många "eliciterade krav" kan falla längst vägkanten.

\subsubsection{Frågeformulär}
Frågeformulär har fördelen att kunna samla in statistisk information om vad /kunder/användare vill ha för system men som tidigare presenterats finns ett ganska fundamentalt problem med denna metod: "Vad händer om den som skapar formuläret och frågorna och den som svarar på frågorna inte har samma referensram?". Min slutsats är att frågeformulär är en bättre metod för att elicitera krav än introspektion då man involverar kunder och användare mer. Men att även den är bristfällig.

\subsubsection{Öppna intervjuer}
Min slutsats är att Öppna intervjuer kan vara en de bättre metoderna för elicitering av krav eftersom kunder och användare är mer engagerade i eliciteringsprocessen. Dock finns det två problem med metoden. Dels samma problem som med frågeformulär (och introspektion) och dels tidsaspekten. att utföra tillräckligt många öppna intervjuer för att identifiera alla relevanta krav kan ta för lång tid och kosta för mycket för ett mindre utvecklingsprojekt.

\subsubsection{Fokusgrupper}
Fokusgrupper har potentialen att vara den bästa metoden för elicitering av krav (så länge en eller flera gruppen, med rätt komposition, kan samlas). Fokusgrupper lider inte av samma problem med tid som öppna intervjuer och den referensram som gruppen har borde vara uppenbar efter att gruppen har träffats. Problemen med fokusgrupper (som presenterats tidigare) och C. Lindes och J. A. Goguens slutsats \cite{goguen1993techniques}, att metodens begränsningar bör studeras, får mig att tveka att rekommendera metoden.

\subsubsection{GREM}
GREM är den metod som är svårast för mig att rekommendera. Utifrån den data som presenterats av P. Lombriser et al. verkar metoden lovande men eftersom metoden är beroende av extern mjukvara för elicitera krav och metoden är så pass ny så kan jag inte komma till slutsatsen att denna metod kan rekomenderas.

\subsubsection{Slutsats}
Alla metoder har sina egna styrkor och svagheter.

Min slutsats är att det bästa tillvägagångssättet för att elicitera krav är att använda flera av de metoder som presenterats för att säkerställa att alla krav som bör ställas på ett system identifieras och specificeras        

\subsection{Vilka fördelar och nackdelar finns med att använda IEEE std 830 i ett programvaruutvecklings projekt av den storleken som detta projekt?}
Min slutsats är at IEEE std 830 ger en bra översikt över vad som bör vara med i en "bra" kravspecifikation, men att dessa fördelar inte väger upp hur lång tid det tar att sätta sig in i standarden för utvecklare i ett mindre projekt. Standarden ger en bra grund att bygga på, det goda skäl att använda den som bas och efter att ha läst igenom standarden så förstår man hur väluttänkt den är. Problemet med standarden är att det tar lång tid att sätta sig in i standarden. Tid som kan vara knapp i början av ett projekt. 

Avslutningsvis så kan jag citera Kristian Sandahl, examinatorn i denna kurs och en man med lång erfarenhet av utveckling, i en mailväxling om standarden med mig, "830 är den mest kompletta beskrivningen av en kravspec jag känner till. Det jag speciellt gillar är att den i flera avseenden nämner att man måste anpassa sin kravspec till hur man ser på det aktuella systemet."   
\subsection{Vilka erfarenheter kan dokumenteras från kravhantering och elicitering av krav i projektet som kan vara intressanta för framtida projekt?}
De erfarenheter som har kunnat dokumenteras kring ämnet kravhantering och elicierting av krav är följande:
\begin{itemize}
	\item Att skapa en ”bra” kravspecifikation är svårt
	\item För att identifiera relevanta krav gäller det att både utvecklingsgruppen och kunden är tillräckligt insatta i det som ska utvecklas. (speciellt om projektets mål är att vidareutveckla ett system.
	\item Att blint följa IEEE std 830 för att skriva en kravspecifikation kan vara frustrerande.
	\item Kunden (och/eller tänkta användare) borde vara mer involverad vid specificering av krav än de var i projektet.
	\item Vid vidareutveckling är det viktigt att testa det befintliga systemet ordentligt innan man specificerar krav.
\end{itemize}


%%%%%%%%%%%%%%%%%%%%%%%%%%%%%%%%%%%%%%%%%%%%%%%%%%%%%%%%%%%%%%%%%%%%%%
%%% Jannering-report.tex ends here
