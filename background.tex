\chapter{Bakgrund}
\label{cha:background}

%% Varför vill kunden utveckla system X?
%% Vilka är era tidigare projekterfarenheter? Förbättringspunkter? %% Bra erfarenhet?

Kunden i detta projekt var Avdelningen för datorseende (CVL), institutionen för systemteknik (ISY) vid Linköpings tekniska högskola. Som tidigare nämnts så är detta projekt en uppföljning på ett projekt som utfördes våren 2016. 
CVL bedriver utbildning och forskning om signalbehandling, bildanalys, datorseende, beräkningsfotografi; detektion, följning och igenkänning av objekt; skattning av pose och 3D-struktur; robotseende och autonoma system; medicinsk bildanalys och bildrekonstruktion. Systemet som användes i projektet var först uppbyggt av en doktorand som använde det i ett forskningsprojekt. Efter detta bestämde CVL att systemet skulle om-användas, och kom med förslaget att använda avståndskameran i kurser relaterade till bildsensorteknik och att utlysa ett kandidatprojekt för att vidareutveckla systemet med en inriktning mot att kunna använda det för att skanna tre dimensionella objekt och sedan skriva ut dem med en 3D-skrivare. Det första kandidatprojektet (våren 2016) hade som mål att dekonstruera rotationsbordet (som var specialbeställt och levererades utan teknisk dokumentation) och sedan implementera styrning av systemet och generering av punktmoln. Efter projektet utlyste CVL ett nytt kandidatprojekt vars mål var att vidareutveckla det befintliga systemet för att till slut kunna använda det för att skriva ut skannade objekt.
CVLs mål med det system som nu är utvecklat är att använda det i forskning kring punktmoln, b.la. vidareutveckling av C++ biblioteket PCL (Point Cloud Library), och annan forskning kring avståndskamerateknik och 3D-skrivarteknik. CVL har även en tanke om att använda systemet för att skapa hinder som kommer att användas i deras nya labb för obemannade farkoster och att konstruera laborationer som använder systemet.
Projektgruppen består (som nämnts tidigare) av studenter vid D (Civilingenjörsutbildning i datateknik) och U (Civilingenjörsutbildning i mjukvaruteknik) programmen vid Linköpings tekniska högskola. Tidigare mjukvaruutvecklingserfarenhet har alla gruppmedlemmar fått genom kurser på respektive program. För de medlemmar som läser D programmet är projektkursen ”Konstruktion med mikrodatorer” den kurs som gett mest erfarenhet av att jobba med ett utvecklingsprojekt i grupp. Kursen gick ut på att, i grupp, konstruera en robot. För de medlemmar som går U programmet är det kursen ”Artificiell intelligens - projekt” som gett mest erfarenhet av att jobba med ett utvecklingsprojekt i grupp. Kursen gick ut på att, i grupp, identifiera relevanta AI-tekniker och literatur som beskriver dem för att sedan utvärdera och jämföra dessa tekniker relativt varandra. Slutligen implementerades och integrerades dessa AI-tekniker i ett valfritt system.

%%%%%%%%%%%%%%%%%%%%%%%%%%%%%%%%%%%%%%%%%%%%%%%%%%%%%%%%%%%%%%%%%%%%%%
%%% background.tex ends here
