\chapter{Bakgrund}
\label{cha:background}

%% Varför vill kunden utveckla system X?
%% Vilka är era tidigare projekterfarenheter? Förbättringspunkter? %% Bra erfarenhet?

Kunden i detta projekt var avdelningen för datorseende (CVL), institutionen för systemteknik (ISY) vid Linköpings universitet. CVL bedriver utbildning och forskning inom dessa områden:
\begin{itemize}
\item signalbehandling
\item bildanalys
\item datorseende
\item beräkningsfotografi
\item detektion, följning och igenkänning av objekt
\item skattning av pose och 3D-struktur
\item robotseende och autonoma system
\item medicinsk bildanalys och bildrekonstruktion.
\end{itemize}
 
Hårdvaran som används i projektet består av ett rotationsbord, en linjärenhet, en avståndskamera samt två datorer för att styra dessa. Detta system byggdes up av doktorand som använde det i ett forskningsprojekt. Efter detta bestämde CVL att systemet skulle användas vidare till kurser i bildsensorteknik. CVL utlyste ett kandidatprojekt för att vidareutveckla systemet för att kunna använda det till att skanna tredimensionella objekt. Detta projekt hade som mål att dekonstruera rotationsbordet (som var specialbeställt och levererades utan teknisk dokumentation) och sedan implementera styrning av systemet och generering av punktmoln. Detta projekt genomfördes våren 2016.

Efter att systemet utvecklats till att skanna tredimensionella objekt, utlyste CVL ett nytt kandidatprojekt. Det är detta projekt som vi genomförde och det är en uppföljning på projektet som genomfördes våren 2016. CVLs mål med det nya kandidatarbetet som nu är utvecklat är att använda det i forskning kring punktmoln, vidare forskning inom Point Cloud Library (PCL) som är ett bibliotek till C++ samt annan forskning kring avståndskamerateknik och 3D-skrivarteknik. CVL har även en tanke om att använda systemet för att skapa hinder som kommer att användas i deras nya labb för obemannade farkoster och att konstruera laborationer som använder systemet.

Projektgruppen består av studenter vid civilingenjörsutbildningen i datateknik och mjukvaruteknik vid Linköpings universitet. Tidigare mjukvaruutvecklingserfarenhet har alla gruppmedlemmar fått genom kurser på respektive program. För de medlemmar som läser datateknik är projektkursen ”Konstruktion med mikrodatorer” den kurs som gett mest erfarenhet av att jobba med ett utvecklingsprojekt i grupp. Kursen gick ut på att konstruera en robot tillsammans i grupper om 6 studenter. För de medlemmar som läser mjukvaruteknik är det kursen ”Artificiell intelligens - projekt” som gett mest erfarenhet av att jobba med ett utvecklingsprojekt i grupp. Kursen gick ut på att identifiera relevanta AI-tekniker och literatur som beskriver dem för att sedan utvärdera och jämföra dessa tekniker relativt varandra. Detta gjordes i grupper om 4-6 personer. Slutligen implementerades och integrerades dessa AI-tekniker i ett valfritt system. Samtliga gruppmedlemmar har tidigare haft dåliga erfarenheter av projekt som styrs med en för lös hand. Gruppen beslutade därför i ett tidigt skede för en strikt uppdelning av roller med förbestämda möten, kommunikationskanaler och aktivitetsuppdelning.

%%%%%%%%%%%%%%%%%%%%%%%%%%%%%%%%%%%%%%%%%%%%%%%%%%%%%%%%%%%%%%%%%%%%%%
%%% background.tex ends here
