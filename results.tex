\chapter{Resultat}
\label{cha:results}

\section{Systembeskrivning}
%% Översiktlig beskrivning system, dokument
%% Vad har kunden för värde hos det som skapats?


\section{Gemensamma erfarenheter}
%% Goda, mindre bra
%% I projektets alla faser
%% Tekniska, process-relaterade


\section{Översikt över individuella bidrag}

This chapter presents the results. Note that the results are presented
factually, striving for objectivity as far as possible.  The results
shall not be analyzed, discussed or evaluated.  This is left for the
discussion chapter.

In case the method chapter has been divided into subheadings such as
pre-study, implementation and evaluation, the result chapter should
have the same sub-headings. This gives a clear structure and makes the
chapter easier to write.

In case results are presented from a process (e.g. an implementation
process), the main decisions made during the process must be clearly
presented and justified. Normally, alternative attempts, etc, have
already been described in the theory chapter, making it possible to
refer to it as part of the justification.

%%%%%%%%%%%%%%%%%%%%%%%%%%%%%%%%%%%%%%%%%%%%%%%%%%%%%%%%%%%%%%%%%%%%%%
%%% results.tex ends here