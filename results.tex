\chapter{Resultat}
\label{cha:results}

\section{Systembeskrivning}
%% Översiktlig beskrivning system, dokument
%% Vad har kunden för värde hos det som skapats?

Systemet består hårdvarumässigt av ett rotationsbord, en avståndskamera och en linjärenhet för att flytta avståndskameran längs en axel. För att styra detta fanns det sedan tidigare mjukvara som kan utföra en enskild skanning. Detta projekt har resulterat i mjukvara för att utföra fler skanningar från olika vinklar och sammanfoga dessa till ett komplett objekt som kan skrivas ut på en 3D-skrivare.

Resultatet av projektet består av en mjukvara som kan användas för att skanna ett fysiskt objekt och skriva ut en kopia av objektet på en 3D-skrivare.

I nuläget finns en väl genomtänkt plan för hur projektet ska genomföras och hur systemet ska fungera. Det finns en projektplan som beskriver hur utvecklingsarbetet ska gå till och en kravspecifikation som säger vad systemet måste uppfylla för att bli godkänt av kunden.

Det har även tagits fram en arkitekturbeskrivning som förklarar hur systemet ska fungera och vilka moduler som ska finnas. Arkitekturen ger en bra utgångspunkt för hur systemet ska implementeras. I arkitekturbeskrivningen finns det förklarat vilka noder systemet ska bestå av och hur de ska kommunicera med varandra. Det underlättar under utvecklingen då varje nod har tydligt definierad in- och utdata och tydligt specificerat vad den ska utföra.

TreeD-wrapper

Insamling av punktmoln - värde för kunden

Mycket bra research

ROS-noder




\section{Gemensamma erfarenheter}
%% Goda, mindre bra
%% I projektets alla faser
%% Tekniska, process-relaterade

Att göra efterforskningar innan man börjar med någonting är väldigt viktigt och det märktes direkt i projektets början. Både genom att större delen av våra efterforskningar kom till stor hjälp direkt i projektet och bara några dagar in i första iterationen märktes det att det fanns några områden vi behövde läsa på mer om innan vi kunde skriva kod. Ett exempel på bra efterforskning som gjordes i förstudien var efterforskningarna kring ROS där alla gruppmedlemmar gick igenom tutorials och läste dokumentation som utvecklingsledaren gått igenom och tagit fram. I slutet av förstudien gjordes en gemensam kodutmaning där alla fick skriva varsin chattklient med hjälp av ROS. Kodutmaningen ledde till att alla kom in i ROS, Git, Python och andra verktyg som skulle användas under resten av projektet.

Betydelsen av kommunikation var stor under projektets gång. Verktygen Slack och Trello har använts flitigt och varit givande för gruppen. Med Slack har vi alltid kunnat nå varandra snabbt och smidigt. Vi har kunnat hålla informationsflöden skilda och sorterade i olika kanaler. Funktioner som påminnelser och trådar har också varit till stor hjälp för att lätt kunna komma åt den informationen som man vill åt i flödena. För att ha en översikt i hur det går med olika delar av projektet har man använt Trello. Genom att ha olika listor för statusen av dokument och features under utveckling har gjort att man som gruppmedlem lätt kunnat se hur det går och vad man behöver göra. 

Kvalité har varit viktigt för gruppen och kvalitetsansvarig har gjort ett bra jobb med att se till att projektets kod och dokument håller en hög standard. Detta framförallt genom granskning av dokument och kod. All kod i projektet har gått genomgått granskning för att säkerställa god kvalité. Till detta har Github's pull request feature använts där alla gruppmedlemmar kunnat kommentera och diskutera koden innan den går in i master branchen hos repositoriet. Dokumenten har också granskats, då genom korrekturläsning. All denna granskning har varit mycket bra för att se hur andra skriver dokument och kod.  

\section{Översikt över individuella bidrag}

Skriva vilka delar alla jobbat på

This chapter presents the results. Note that the results are presented
factually, striving for objectivity as far as possible.  The results
shall not be analyzed, discussed or evaluated.  This is left for the
discussion chapter.

In case the method chapter has been divided into subheadings such as
pre-study, implementation and evaluation, the result chapter should
have the same sub-headings. This gives a clear structure and makes the
chapter easier to write.

In case results are presented from a process (e.g. an implementation
process), the main decisions made during the process must be clearly
presented and justified. Normally, alternative attempts, etc, have
already been described in the theory chapter, making it possible to
refer to it as part of the justification.

%%%%%%%%%%%%%%%%%%%%%%%%%%%%%%%%%%%%%%%%%%%%%%%%%%%%%%%%%%%%%%%%%%%%%%
%%% results.tex ends here